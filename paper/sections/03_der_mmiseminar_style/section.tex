\section{Der mmiSeminar-Style}
\label{sec:CSstyle}

Der zunächst von Peter Dörfler für die Seminare am Lehrstuhl für Technische Informatik 
erstellte Style wird mit dem Wintersemester 06/07 auch für die Seminare des Instituts 
für Mensch-Maschine-Interaktion eingesetzt. Der Style bietet einige Makros, die das 
Erstellen von LaTeX Dokumenten vereinfachen. Durch Ändern einiger Verhaltensweisen 
von LaTeX wird es möglich, den Seminarband als ein TeX-Dokument zu behandeln. 
Weiterhin sind darin einheitliche Formatierungen und das Layout für den Text definiert.

\subsection{Die Titelseite}

\label{sub:CStitle}

Die Titelseite lässt sich recht schnell erstellen (siehe Quellcode). Es müssen
nur die folgenden Werte (eigentlich Makros) gesetzt werden:

\begin{narrowitems}
\item Der Autor des Textes
\item Der Titel
\item Die e-mail des Autors
\item Zusammenfassung
\item Schlüsselworte für den Inhalt
\end{narrowitems}
Der Befehl \verb+\makeArticleTitle+ erstellt dann die Titelseite. 

\subsection{Listen-Umgebungen im MMI-Style}
\label{sub:CSmmiLists}

\subsubsection{Aufzählungen}
Zwei Arten von Aufzählungen sind hinzugekommen:
\begin{narrowenum}
  \item \texttt{narrowenum}
  \item \texttt{narrowitems}
\end{narrowenum}
Die erste der beiden hat direkt hier Anwendung gefunden. Der andere
Aufzählungstyp ist in Abschnitt \ref{sub:CStitle} zu sehen. Diese Umgebungen
sind für Fälle gedacht, in denen die einzelnen Punkte sehr kurz sind. Die
normalen Abstände wirken einfach zu groß, wenn pro Punkt nur zwei, drei
Worte stehen.

\subsubsection{Algorithmus-Umgebung}

Die \texttt{algorithm}-Umgebung dient dazu Algorithmen
darzustellen. Sie funktioniert im Prinzip wie andere Listenumgebungen,
hat aber ein etwas anderes Layout. Zusätzlich sind alle weiteren Stufen nach
der ersten in die \texttt{subalgorithm}-Umgebung zu setzen. Üblicherweise wird
ein Algorithmus in eine \emph{floating}-Umgebung gesetzt. Das heisst, dass er
nicht unbedingt an der Stelle erscheint, an der der Algorithmus im Text steht, 
sondern eventuell in den nächsten freien Platz geschoben wird.

\begin{figure}[htb]
  \begin{algorithm}
  \item begin begin begin begin begin begin begin begin begin begin begin 
  			begin begin begin begin begin begin begin 
  \item \begin{subalgorithm}
    \item Item21 Item21 Item21 Item21 Item21 Item21 Item21 
    \item Item22 Item22 Item22 Item22 Item22 Item22 Item22 
    \end{subalgorithm}
  \item end
  \end{algorithm}
  \caption{Ein Beispiel-Algorithmus}
  \label{fig:CSalgo}
\end{figure}
Die \texttt{figure}-Umgebung (siehe Quellcode) ist für floating Bilder, mit der
Position als Parameter (\textbf{h}ere, \textbf{t}op, \textbf{b}ottom). Der
Befehl \texttt{caption} legt die Bildunterschrift fest. Diese Umgebung gehört
zu Standard-Latex.

\subsection{Bilder und Tabellen}
\label{sub:CSfigtab}

Sowohl für Tabellen als auch für Abbildungen gibt es im mmiSeminar-Style
floating-Umgebungen, die einfacher in der Anwendung sind als die
Standardumgebungen. 

\subsubsection{Abbildungen}

Abbildungen sollten im .jpg, .png oder .gif Format vorliegen und in der  \emph{figure}-Umgebung per \emph{includegraphics}-Befehl eingebunden werden. Die Bildgröße kann dabei auch relativ zur Textbreite angegeben werden, wie in Bild \ref{fig:Testbild} gezeigt.

\begin{figure}[htb]
 \centering \includegraphics[width=0.5\textwidth]{bilder/TestBild.png}
 \caption{Bildunterschrift für das Testbild. Sollte das Bild von einer anderen Quelle stammen wird hier auch mit cite darauf verwiesen.}
 \label{fig:Testbild}
\end{figure}

Auch in Abbildungen sollte es natürlich möglich sein, allen Text ohne Probleme
zu lesen. Leider sieht man oft das Gegenteil.

\subsubsection{Tabellen}

Die Tabellen-Umgebung \texttt{vtable} ist so ähnlich aufbebaut wie die
\texttt{xxxpsfig} Befehle weiter oben. Die Parameter sind in Reihenfolge:
\begin{narrowenum}
  \item Position
  \item Label
  \item Caption
  \item eigentliche Tabelle
\end{narrowenum}
Für die Labels gilt das gleich wie schon bei den Abbildungen. Der
Referenzierungsbefehl ist \texttt{tref}.  Die eigentliche Tabelle ist am
besten an einem Beispiel, Tabelle \tref{figs}, erklärt. Direkt nach dem Beginn
der \texttt{tabular}-Umgebung folgt ein Format-String, der die Ausrichtung des
Textes in den einzelnen Spalten festlegt. Die Optionen \texttt{l}, \texttt{r}
und \texttt{c} stehen für links, rechts und mittig ausgerichtet. Mit
\texttt{p\{Länge\}} kann man eine Spalte fester Breite deklarieren. Der
senkrechte Strich erzeugt einen ebensolchen. Vom Layout gefälliger und auch
lesbarer ist die Verwendung nur nötiger Linien. Die im Beispiel wäre \zB\ auch
nicht nötig gewesen. Mit \texttt{multiline\{spalten\}\{ausrichtung\}\{text\}}
kann man einen Text über mehrere Spalten ziehen und ausrichten wie angegeben.
Alles weitere steht in der angegebenen Literatur.

\appendixOfArticle
