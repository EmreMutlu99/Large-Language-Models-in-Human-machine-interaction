\section{Sehr kurze Einführung in \LaTeX}

\label{sec:CSlatex}

Für komplexere Aufgaben sind weitere Nachschlagewerke unverzichtbar. 
Eine recht umfassende Beschreibung ist kostenlos erhältlich:
\glqq\emph{The Not So Short Introduction to \LaTeX2e}\grqq \cite{oetiker},
eine Einführung in Latex. Für schwierige Formeln ist das \AmS-\LaTeX-Package
sehr zu empfehlen. Das \glqq\emph{\AmS-\LaTeX\ User's Guide} \cite{ams}
beschreibt diese häufig verwendete Erweiterung zu LaTex. Die beiden Dokumente
sind über die Seminar-Seite
(\webaddress{http://www.mmi.rwth-aachen.de/Lehre/Seminare/Download})
verlinkt.  Weiterhin gibt es einige durchaus empfehlenswerte Bücher: 
\glqq\emph{\LaTeX: A Document Preparation System}\grqq, das Reference Manual \cite{latexManual} 
und \glqq\emph{The \LaTeX\ Companion}\grqq \cite{latexCompanion}, Erklärungen zu verschiedenen
Zusatzpackages u.\,a.\ \AmS-\LaTeX. Ausserdem gibt es noch die drei Latex
Bücher von Kopka, von denen nur das erste \glqq\emph{\LaTeX\ Einführung}\grqq
\cite{kopka1} für den einfachen Anwender von LaTex nötig ist.

\subsection{Überschriften und Referenzen}

\label{sub:CSheadlines}

Wie in diesem Beispieldokument zu sehen ist, gibt es für Artikel drei
Hierarchieebenen der Überschriften. Sie werden mit den Befehlen
\verb+\section{...}+, \verb+\subsection{...}+ und \verb+\subsubsection{...}+
gesetzt. Dabei ist \texttt{...} durch die jeweilig Überschrift zu
ersetzen. Innerhalb jedes Blocks kann durch den Befehl \texttt{label} eine
Marke für spätere Referenzierung gesetzt werden. In allen sochen Marken sollten
wiederum (wie bei Bilddateien) am Anfang stehen. Für diesen Abschnitt ist im
Quellcode der Befehl \verb+\label{sec:CSlatex}+ zu finden. Mit dem Befehl
\verb+\ref{sec:CSlatex}+ wird die Abschnittsnummer dieses Abschnitts
ausgegeben: 2. Für jede Überschriftenebene, sowie Formeln, Aufzählungspunkte,
Definitionen usw.\ gibt es eigene Zähler, die entsprechen mit label gesetzt
und mit ref wieder abgerufen werden können (s. Quelltext). Es zählt jeweils
die unterste Stufe. Daher ist es sinnvoll, den label Befehl direkt nach einem
einen neuen Block einleitenden Befehl zu verwenden.

Absätze werden in LaTeX durch eine oder mehrere Leerzeilen erreicht. Alle
überflüssigen Leerzeichen und -zeilen werden ignoriert. Befehle wie \verb+\\+
(newline), \verb+\clearpage+ oder \verb+\newpage+ sind nur in Sonderfällen zu
verwenden und in dieser Arbeit gar nicht.

\subsection{Textformatierung}
\label{sub:CSformat}

LaTeX bietet einige Möglichkeiten, die Schriftart zu verändern. Diese sollten
sparsam eingesetzt werden, um die Lesbarkeit des Textes nicht zu
beeinträchtigen. \emph{Wichtige} oder \emph{neue} Begriffe werden betont
(engl. emphasize) mit dem Befehl \verb+\emph{...}+. Die
Schreibmaschinenenschrift \verb+\texttt{...}+ wird für Befehle o\,ä.\ 
verwendet. Bei Befehlen, die LaTeX-Steuerzeichen enthalten empfiehlt sich der
Befehl \verb \verb+...+ \ (vgl. Kommentar im Quellcode in Abschnitt
\ref{sub:CStools}).

\subsection{Aufzählungen}
\label{sub:CSlists}

LaTeX stellt im wesentlichen zwei Arten von Aufzählungen zur Verfügung:
nummerierte und unnummerierte. Es handelt sich dabei um so genannte Umgebungen
(engl. environments), die immer von \verb+\begin{env}+ und \verb+\end{env}+
umschlossen werden. Dabei ist \texttt{env} durch den Namen der jeweiligen
Umgebung zu ersetzen. Wer Emacs verwendet kann mit \texttt{Ctrl-c Ctrl-e}
leicht Environments erstellen. Eine spezifische Aufzählungsumgebung für
Algorithmen wird in Abschnitt \ref{sub:CSmmiLists} erläutert. Es folgen
zwei Beispiele zu den beiden Aufzählungsarten.

\subsubsection{Itemize-Umgebung}
Dient der nicht nummerierten Aufzählung. Beispiel:
\begin{itemize}
\item Item1
\item Item2
\item Item3
\end{itemize}

\subsubsection{Enumerate-Umgebung}
Dient der nummerierten Aufzählung. Beispiel:
\begin{enumerate}
\item Item1
\item Item2
\item Item3 \label{enum:CSsubenum}
  \begin{enumerate}
  \item Item31
  \item Item32
  \item Item33
  \end{enumerate}
\end{enumerate}
Der \texttt{label}-Befehl funktioniert auch innerhalb von Aufzählungen 
(siehe Quellcode): Punkt \ref{enum:CSsubenum} ist in mehrere Unterpunkte aufgeteilt.

\subsection{Formel-Darstellung}
\label{sub:CSmath}

LaTex bietet zahlreiche komfortable Möglichkeiten, um Formeln darzustellen. 
Da aber die Darstellung sehr komplexer Formeln ermöglicht wird, existieren entsprechend viele 
Befehle zu ihrer Beschreibung. Zunächst muss man zwischen mathematischen
Ausdrücken im Text, wie $\alpha\in\mathbb{R}$, und abgesetzten Formeln oder
Ausdrücken entscheiden. Es folgen einige Beispiel aus anderen Texten, die
unter Zuhilfenahme des Quellcodes den Umgang mit Formeln verdeutlichen
sollten.

Hier zunächst zwei einfache Formeln.

\begin{equation}
  h_i(z)=\frac{1}{\sqrt{2\pi}\sigma_i}\cdot e^{-\left(\frac{z^2}{2\sigma_i^2}\right)}
\end{equation}
\begin{equation}
o_j=f_{\mathrm{sig}} \left( \sum_{i=1}^N c_{ij} h(||x-w_i||) \right)
\end{equation}

Mit \verb+\intertext{...}+ lässt sich Text innerhalb ausgerichteter
Formelblöcke einfügen.

\begin{align}
    w_i(t+1) &= w_i(t)-\alpha(t)\left[x(t)-w_i(t)\right]\nonumber \\ 
    w_j(t+1) &= w_j(t)+\alpha(t)\left[x(t)-w_j(t)\right]
  \label{eq:CSlvq2}
  \intertext{where $w_i$ must be of a different class than $x$ and $w_j$ of
    the same class. If the two closest prototypes are of the same class
    (training of type (2) the following modification scheme is applicable (no
    "window"\ constraint):}
  w_k(t+1) &= w_k(t) + \epsilon\alpha(t)\left[ x(t) - w_k(t)\right]
  \label{eq:CSlvq3}
\end{align}

\begin{equation}
\begin{split}
\label{eq:CScov}
\Sigma &= E\{(\mathbf{\mathbf{x}}-\mathbf{m})(\mathbf{\mathbf{x}}-\mathbf{m})^T\} \\
&= \left[ \begin{array}{ccc}
    E\{(x_1-m_1)(x_1-m_1)\} & \cdots & E\{(x_1-m_1)(x_n-m_n)\}\\
    \vdots & & \vdots \\
    E\{(x_n-m_n)(x_1-m_1)\} & \cdots & E\{(x_n-m_n)(x_n-m_n)\}\\
       \end{array} \right] \\
     &= \left[ \begin{array}{ccc}
         c_{11} & \cdots & c_{1n} \\
         \vdots & & \vdots \\
         c_{n1} & \cdots & c_{nn}
       \end{array} \right]
\end{split}
\end{equation}
Auch Formeln können natürlich Label besitzen, \zB\: Die Modifikation der Gewichte
beim LVQ3 Training \eqref{eq:CSlvq3} versteht ohne Erklärung keiner.


