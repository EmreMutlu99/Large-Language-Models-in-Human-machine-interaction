\selectlanguage{ngerman} %choose: ngerman or English
%%%%% Angaben für die Titelseite

% Titel der Arbeit
\Vtitle{Seminar MMI - Vorlage für \\Schriftliche Ausarbeitungen}
% Autor der Arbeit
\Vauthor{Ömer Emre Mutlu}
% e-mail des Autors
\Vaddress{emre.mutlu@rwth-aachen.de}
% Kurzfassung der Arbeit
\Vabstract{
Dieses Dokument dient als Anleitung und als Vorlage für die schriftlichen 
Ausarbeitungen für Seminare am \emph{Institut für Mensch-Maschine-Interaktion 
(MMI)}. Einsteiger in \LaTeX\ sollten parallel mit einem Ausdruck dieser
Vorlage und dem Quellcode arbeiten, um so leicht entspechende Befehle für
ein gewünschtes Layout zu finden. Helfen dieses Dokument sowie angegebene
Literatur nicht weiter, stehen natürlich die Betreuer für Fragen bereit. 
}
% Schlüsselworte zum Inhalt der Arbeit
\Vkeywords{\LaTeX, mmiSeminar-Style, Vorlage}
% tatsächliche Erstellung der Titelseite
\makeArticleTitle


% Beispiel für einen Abschnitt

\section{Einführung}
\label{sec:CSorg}

% Beispiel für Fußnoten
% Beispiel für Literaturangaben

\LaTeX\ basiert auf \TeX\footnote{sprich: Tech}, einem Textsatzsystem, das von 
Donald E. Knuth \cite{knuth} entwickelt und veröffentlicht wurde, um wissenschaftlichen 
Texten ein einheitliches und gutes Layout zu geben. Aufgrund seiner komplexen Syntax 
ist die direkte Verwendung von \TeX\ jedoch nicht leicht, deshalb hat Leslie Lamport 
\cite{latexManual} das Makro-Paket \LaTeX\ entwickelt. Dieses definiert Makros, 
die es dem Benutzer erlauben, komplexere und häufig benötigte Folgen von \TeX-Kommandos 
zu verwenden. Die neuste und vereinheitlichte Version ist \LaTeXe\, das wir hier verwenden.

% Beispiel für einen Unter-Abschnitt

\subsection{Warum \LaTeX}

% Beispiel für Hervorhebungen
% Beispiel für Sonderkommando \zb

\LaTeX\ (wegen besserer Lesbarkeit auch einfach LaTeX) ist eine Makro-Sprache 
und muss ähnlich einer Programmiersprache erlernt werden. Entsprechend muss 
TeX-Code erst compiliert oder "`\emph{getext}"' werden, bevor das tatsächliche 
Layout sichtbar wird. Doch die Vorteile der Verwendung von LaTeX für professionelle 
wissenschaftliche Texte treten besonders bei längeren (etwa 100 Seiten) Texten 
mit vielen Bildern und Formeln zutage. LaTeX setzt Grundregeln um, an denen sich auch 
professionelle Layouter orientieren, um einen Text gut lesbar und optisch ansprechend 
zu gestalten. Zusätzlich wird durch die festen Befehle (\zB\ für Kapitelanfänge) ein hohes Maß 
an Konsistenz erreicht. Arbeiten im technischen Bereich werden daher häufig in TeX ausgeführt.
Zum Beispiel verlangen viele Institute verlangen, dass Studien- und Diplomarbeiten in LaTeX geschrieben
sind. Fast alle internationalen Zeitschriften und Konferenzen erwarten TeX-Dokumente, 
für die sie entsprechende eigene \emph{Styles} entwickelt haben.

\subsection{Vom Quelltext zum PostScript}

\subsubsection{Dateien rund um \LaTeX}

% Beispiel für Schreibmaschinen-Schriftart

Der Quellcode für jedes LaTeX-Dokument steht in einer \emph{.tex}-Datei. 
Mit dem Befehl \texttt{input} können weitere Dateien eingebunden werden.
Dies hat den gleichen Effekt, als ob der Quellcode an Stelle des input-Befehls 
stände\footnote{siehe auch Datei \texttt{frame.tex}}. Befehle in LaTeX beginnen 
übrigens immer mit einem backslash $\backslash$.

In wissenschaftlichen Arbeiten werden sehr häufig Ergebnisse anderer Forscher 
zitiert. Im technischen Bereich geschieht dies durch eine Abkürzung oder 
Indexnummer, die im \emph{Literaturverzeichnis} wiederzufinden ist. 
Dort stehen dann genaue Angaben zu der jeweiligen Veröffentlichung. 
In LaTeX gibt es ein einheitliches Format in dem die Daten einer Veröffentlichung, 
wie \zB\ Autor, Titel, Jahr, usw.\ gespeichert werden. Mittels verschiedener 
Bibliography-Styles kann das Layout des Literaturverzeichnisses leicht verändert werden. 
Die Angaben zu den zitierten Arbeiten stehen in der Bibliography (\emph{.bib}-Datei),
die dann von BiBTeX ausgewertet wird.

Zur Umsetzung verschiedener Layouts im Text wird ein ähnliches Vorgehen wie
für das Literaturverzeichnis angewandt. Zunächst gibt es die \emph{Dokumentenklasse} 
(Befehl \texttt{documentclass} in frame.tex), die die grundsätzliche Formatierung festlegt. 
Dann erweitern die Styles (in \emph{.sty}-Dateien) die Dokumentenklassen 
und bieten weitere Makros an, die die Arbeit mit LaTeX vereinfachen sollen. 
Ein Beispiel ist iso8859-1.sty: Ohne diesen Style kann man Umlaute nicht direkt in den Text schreiben. 
Der mmiSeminar-Style dagegen dient der einheitlichen Formatierung und einfachen Erstellung der 
schriftlichen Ausarbeitungen zur späteren Zusammenführung in einem Seminarband.

\subsubsection{Von \LaTeX\ zu PostScript}

% Beispiel für Gedankenstriche

Der erste Schritt um aus dem geschriebenen Quellcode eine formatierte
PostScript-Version zu erstellen ist das so genannte "`\emph{Texen}"' --- das
Ausführen des \texttt{latex} Befehls. Dies geht entweder über die Konsole mit
Angabe des Dateinamens der Master-Datei (hier: \texttt{frame}), oder per
Menüpunkt im verwendeten Editor. In diesem Schritt werden einige Hilfsdateien
mit gleichem Namen wie die Master-Datei erstellt. Der eigentliche Text steht
in der \emph{.dvi}-Datei (dvi steht für device independent). 
Unter Linux und unter Windows existieren Programme zur Anzeige dieser Dateien.

Je nachdem, welche Änderungen im Quellcode vorgenommen wurden, muss \texttt{latex} bis
zu dreimal ausgeführt werden. Um alle Verweise zu finden, erstellt das Programm
zunächst Tabellen, die im nächsten Schritt verwendet werden. Beim ersten Texen
wird auch eine Liste der Literaturverweise (\texttt{cite}) erstellt. Wurden
die .bib-Datei verändert muss auch der Befehl \texttt{bibtex Master-Datei}
ausgeführt werden. Hierauf folgen in der Regel zwei Aufrufe von \texttt{latex}, 
bis alle Referenzen korrekt sind. Komfortable LaTeX-Entwicklungsumgebungen unter Linux 
und unter Windows führen die notwendigen Schritte auf Knopfdruck automatisch aus.

Aus der nun korrekten .dvi-Datei wird eine PostScript-Datei (.ps) mit dem
Befehl \texttt{dvips -t a4 Master-File} erzeugt. Der Parameter \texttt{-t}
gibt die Papiergröße A4 an. Die Endung des Master-Files ist wieder
wegzulassen. Das Ergebnis kann unter Linux mit \texttt{gv} oder unter 
Windows mit \texttt{ghostview} betrachtet werden.

% Die tabular-Umgebung eignet sich nicht nur für Tabellen, sondern auch zum Anordnen anderer Textteile

Für das Seminar ist die Befehlsfolge, die bei korrektem Quellcode immer zu
einem PostScript führen sollte:
{\tt
  \begin{center}
    \begin{tabular}[c]{ll}
      latex & frame \\
      bibtex & frame \\
      latex & frame \\
      latex & frame \\
    \end{tabular}
  \end{center}
}

\subsection{Tools}
\label{sub:CStools}

Die für Linux und Windows zur Verfügung stehenden Hilfsmittel unterscheiden
sich, obwohl viele der Linux-üblichen Programme ebenso für Windows erhältlich sind. 
LaTeX kommt aus der Unix-Welt und daher ist hier etwas leichter zu verwenden, 
insbesondere sind die benötigten Programme bereits installiert. 
Aber auch unter Windows sind alle Anwendungen, die zum Arbeiten mit LaTeX
gebraucht werden, kostenlos verfügbar. Die jeweiligen Werkzeuge sind in 
den nächsten Abschnitten kurz beschrieben.

\subsubsection{Unter Linux}

Die Programmepakete LaTeX, Emacs und dvips müssen installiert werden.
Emacs bietet für praktisch alle unter Linux üblichen Dateiformate Syntax-Highlighting 
und besondere Makros, die das Programmieren vereinfachen.
Dies trifft auch für .tex und .bib-Dateien zu. Die Kombination \texttt{Ctrl-C
  Ctrl-E} erzeugt \zB\ beliebige LaTeX-Umgebungen. Für BiBTeX erscheint ein
Menüpunkt, mit dem sich neue Einträge für das Literaturverzeichnis generieren
lassen. PostScript-Dateien können mit den kostenlos in der Distribution
enthaltenen Programmen ghostview und gv geöffnet werden.
Grafiken können unter Linux mit xfig erstellt werden. 

% Beispiel für Web-Adresse

\subsubsection{Unter Windows}

Alle Routinen, die für die Verwendung von LaTeX nötig sind, werden unter
Windows im MikTeX-Paket geliefert (siehe
\webaddress{http://www.mmi.rwth-aachen.de/Lehre/Seminare/Download}). Zur
Eingabe des Quellcodes kann prinzipiell jeder Editor verwendet werden, 
mit dem ASCII-Dateien geschrieben werden können. Empfehlenswert ist aber 
die Verwendung spezieller Entwicklungsumgebungen wie zum Beispiel TexnicCenter 
(siehe Downloads), die die beschriebenen Abläufe automatisieren. Viewer,
wie Yap (zur Anzeige von dvi-Formaten) oder GhostView (zur Anzeige von 
PostScript-Dateien) werden getrennt installiert, stehen dann allerdings
in die Entwicklungsumgebungen integriert zur Verfügung.

\subsubsection{Grafiken in LaTeX-Dokumenten}

Grundsätzlich sollen Grafiken für diese Ausarbetung als JPG/PNG/GIF eingefügt werden. Es vorteilhaft, die Bilder direkt auf die benötigte Größe zu bringen.
Maximale Breite ist die Textbreite von 13,7\,cm. Die maximale sinnvolle
Auflösung für Biler ist 150\,dpi. Höhere Auflösungen gehen beim Kopiervorgang
verloren. Die Bildgröße kann außerdem als Faktor zur textbreite angegeben werden.

% Aha, ein Trick! Ich will einen Befehl in Schreibmaschinenschrift schreiben, 
% aber es sind zuviele Sonderzeichen ( \ {} ) darin für \texttt.
% Also nehme ich den  \verb (wie verbatim) Befehl. Zusätzlich nutze ich,
% dass statt der geschweiften Klammern bei \verb im Prinzip ein beliebiges
% Zeichen, das kein Buchstabe ist verwendet werden kann. Dieses darf logischer
% Weise nicht innerhalb der "Klammer" vorkommen. Üblich sind + ! *  

Das Einscannen von Bildern sollte soweit wie möglich vermieden werden. Zum
einen ist die Qualität meist nicht gut und die Dateien sehr groß. Zum anderen
ist das Verwenden von Teilen aus Büchern selten gestattet. Optimal ist eine 
Nachahmung der Abbildung mit Verweis auf das Original: \verb+nach \cite{OriginalAutor}+. 
Sollte dies nicht gehen, kann aus Artikeln das Bild oft mit Grafikprogrammen 
extrahiert werden. Bei Scans sollte darauf geachtet werden, dass das Rauschen möglichst klein ist, 
\ie\ ein weisser Hintergrund sollte keine schwarzen Pünktchen und durchscheinende Seiten enthalten.

\subsection{Organisation der Dateien}

% Beispiel für eine Aufzählung

Um diesen Text texen zu können sind eine Reihe von Dateien nötig. Ihre
Aufgaben sind im Folgenden kurz beschrieben:

\begin{itemize}
\item \texttt{frame.tex}\\
  Dient als Ersatz für die Master-Datei des gesamten Seminarbandes. Der
  Befehl \verb+\selectlanguage{ngerman} %choose: ngerman or English
%%%%% Angaben für die Titelseite

% Titel der Arbeit
\Vtitle{Seminar MMI - Vorlage für \\Schriftliche Ausarbeitungen}
% Autor der Arbeit
\Vauthor{Ömer Emre Mutlu}
% e-mail des Autors
\Vaddress{emre.mutlu@rwth-aachen.de}
% Kurzfassung der Arbeit
\Vabstract{
Dieses Dokument dient als Anleitung und als Vorlage für die schriftlichen 
Ausarbeitungen für Seminare am \emph{Institut für Mensch-Maschine-Interaktion 
(MMI)}. Einsteiger in \LaTeX\ sollten parallel mit einem Ausdruck dieser
Vorlage und dem Quellcode arbeiten, um so leicht entspechende Befehle für
ein gewünschtes Layout zu finden. Helfen dieses Dokument sowie angegebene
Literatur nicht weiter, stehen natürlich die Betreuer für Fragen bereit. 
}
% Schlüsselworte zum Inhalt der Arbeit
\Vkeywords{\LaTeX, mmiSeminar-Style, Vorlage}
% tatsächliche Erstellung der Titelseite
\makeArticleTitle


% Beispiel für einen Abschnitt

\section{Einführung}
\label{sec:CSorg}

% Beispiel für Fußnoten
% Beispiel für Literaturangaben

\LaTeX\ basiert auf \TeX\footnote{sprich: Tech}, einem Textsatzsystem, das von 
Donald E. Knuth \cite{knuth} entwickelt und veröffentlicht wurde, um wissenschaftlichen 
Texten ein einheitliches und gutes Layout zu geben. Aufgrund seiner komplexen Syntax 
ist die direkte Verwendung von \TeX\ jedoch nicht leicht, deshalb hat Leslie Lamport 
\cite{latexManual} das Makro-Paket \LaTeX\ entwickelt. Dieses definiert Makros, 
die es dem Benutzer erlauben, komplexere und häufig benötigte Folgen von \TeX-Kommandos 
zu verwenden. Die neuste und vereinheitlichte Version ist \LaTeXe\, das wir hier verwenden.

% Beispiel für einen Unter-Abschnitt

\subsection{Warum \LaTeX}

% Beispiel für Hervorhebungen
% Beispiel für Sonderkommando \zb

\LaTeX\ (wegen besserer Lesbarkeit auch einfach LaTeX) ist eine Makro-Sprache 
und muss ähnlich einer Programmiersprache erlernt werden. Entsprechend muss 
TeX-Code erst compiliert oder "`\emph{getext}"' werden, bevor das tatsächliche 
Layout sichtbar wird. Doch die Vorteile der Verwendung von LaTeX für professionelle 
wissenschaftliche Texte treten besonders bei längeren (etwa 100 Seiten) Texten 
mit vielen Bildern und Formeln zutage. LaTeX setzt Grundregeln um, an denen sich auch 
professionelle Layouter orientieren, um einen Text gut lesbar und optisch ansprechend 
zu gestalten. Zusätzlich wird durch die festen Befehle (\zB\ für Kapitelanfänge) ein hohes Maß 
an Konsistenz erreicht. Arbeiten im technischen Bereich werden daher häufig in TeX ausgeführt.
Zum Beispiel verlangen viele Institute verlangen, dass Studien- und Diplomarbeiten in LaTeX geschrieben
sind. Fast alle internationalen Zeitschriften und Konferenzen erwarten TeX-Dokumente, 
für die sie entsprechende eigene \emph{Styles} entwickelt haben.

\subsection{Vom Quelltext zum PostScript}

\subsubsection{Dateien rund um \LaTeX}

% Beispiel für Schreibmaschinen-Schriftart

Der Quellcode für jedes LaTeX-Dokument steht in einer \emph{.tex}-Datei. 
Mit dem Befehl \texttt{input} können weitere Dateien eingebunden werden.
Dies hat den gleichen Effekt, als ob der Quellcode an Stelle des input-Befehls 
stände\footnote{siehe auch Datei \texttt{frame.tex}}. Befehle in LaTeX beginnen 
übrigens immer mit einem backslash $\backslash$.

In wissenschaftlichen Arbeiten werden sehr häufig Ergebnisse anderer Forscher 
zitiert. Im technischen Bereich geschieht dies durch eine Abkürzung oder 
Indexnummer, die im \emph{Literaturverzeichnis} wiederzufinden ist. 
Dort stehen dann genaue Angaben zu der jeweiligen Veröffentlichung. 
In LaTeX gibt es ein einheitliches Format in dem die Daten einer Veröffentlichung, 
wie \zB\ Autor, Titel, Jahr, usw.\ gespeichert werden. Mittels verschiedener 
Bibliography-Styles kann das Layout des Literaturverzeichnisses leicht verändert werden. 
Die Angaben zu den zitierten Arbeiten stehen in der Bibliography (\emph{.bib}-Datei),
die dann von BiBTeX ausgewertet wird.

Zur Umsetzung verschiedener Layouts im Text wird ein ähnliches Vorgehen wie
für das Literaturverzeichnis angewandt. Zunächst gibt es die \emph{Dokumentenklasse} 
(Befehl \texttt{documentclass} in frame.tex), die die grundsätzliche Formatierung festlegt. 
Dann erweitern die Styles (in \emph{.sty}-Dateien) die Dokumentenklassen 
und bieten weitere Makros an, die die Arbeit mit LaTeX vereinfachen sollen. 
Ein Beispiel ist iso8859-1.sty: Ohne diesen Style kann man Umlaute nicht direkt in den Text schreiben. 
Der mmiSeminar-Style dagegen dient der einheitlichen Formatierung und einfachen Erstellung der 
schriftlichen Ausarbeitungen zur späteren Zusammenführung in einem Seminarband.

\subsubsection{Von \LaTeX\ zu PostScript}

% Beispiel für Gedankenstriche

Der erste Schritt um aus dem geschriebenen Quellcode eine formatierte
PostScript-Version zu erstellen ist das so genannte "`\emph{Texen}"' --- das
Ausführen des \texttt{latex} Befehls. Dies geht entweder über die Konsole mit
Angabe des Dateinamens der Master-Datei (hier: \texttt{frame}), oder per
Menüpunkt im verwendeten Editor. In diesem Schritt werden einige Hilfsdateien
mit gleichem Namen wie die Master-Datei erstellt. Der eigentliche Text steht
in der \emph{.dvi}-Datei (dvi steht für device independent). 
Unter Linux und unter Windows existieren Programme zur Anzeige dieser Dateien.

Je nachdem, welche Änderungen im Quellcode vorgenommen wurden, muss \texttt{latex} bis
zu dreimal ausgeführt werden. Um alle Verweise zu finden, erstellt das Programm
zunächst Tabellen, die im nächsten Schritt verwendet werden. Beim ersten Texen
wird auch eine Liste der Literaturverweise (\texttt{cite}) erstellt. Wurden
die .bib-Datei verändert muss auch der Befehl \texttt{bibtex Master-Datei}
ausgeführt werden. Hierauf folgen in der Regel zwei Aufrufe von \texttt{latex}, 
bis alle Referenzen korrekt sind. Komfortable LaTeX-Entwicklungsumgebungen unter Linux 
und unter Windows führen die notwendigen Schritte auf Knopfdruck automatisch aus.

Aus der nun korrekten .dvi-Datei wird eine PostScript-Datei (.ps) mit dem
Befehl \texttt{dvips -t a4 Master-File} erzeugt. Der Parameter \texttt{-t}
gibt die Papiergröße A4 an. Die Endung des Master-Files ist wieder
wegzulassen. Das Ergebnis kann unter Linux mit \texttt{gv} oder unter 
Windows mit \texttt{ghostview} betrachtet werden.

% Die tabular-Umgebung eignet sich nicht nur für Tabellen, sondern auch zum Anordnen anderer Textteile

Für das Seminar ist die Befehlsfolge, die bei korrektem Quellcode immer zu
einem PostScript führen sollte:
{\tt
  \begin{center}
    \begin{tabular}[c]{ll}
      latex & frame \\
      bibtex & frame \\
      latex & frame \\
      latex & frame \\
    \end{tabular}
  \end{center}
}

\subsection{Tools}
\label{sub:CStools}

Die für Linux und Windows zur Verfügung stehenden Hilfsmittel unterscheiden
sich, obwohl viele der Linux-üblichen Programme ebenso für Windows erhältlich sind. 
LaTeX kommt aus der Unix-Welt und daher ist hier etwas leichter zu verwenden, 
insbesondere sind die benötigten Programme bereits installiert. 
Aber auch unter Windows sind alle Anwendungen, die zum Arbeiten mit LaTeX
gebraucht werden, kostenlos verfügbar. Die jeweiligen Werkzeuge sind in 
den nächsten Abschnitten kurz beschrieben.

\subsubsection{Unter Linux}

Die Programmepakete LaTeX, Emacs und dvips müssen installiert werden.
Emacs bietet für praktisch alle unter Linux üblichen Dateiformate Syntax-Highlighting 
und besondere Makros, die das Programmieren vereinfachen.
Dies trifft auch für .tex und .bib-Dateien zu. Die Kombination \texttt{Ctrl-C
  Ctrl-E} erzeugt \zB\ beliebige LaTeX-Umgebungen. Für BiBTeX erscheint ein
Menüpunkt, mit dem sich neue Einträge für das Literaturverzeichnis generieren
lassen. PostScript-Dateien können mit den kostenlos in der Distribution
enthaltenen Programmen ghostview und gv geöffnet werden.
Grafiken können unter Linux mit xfig erstellt werden. 

% Beispiel für Web-Adresse

\subsubsection{Unter Windows}

Alle Routinen, die für die Verwendung von LaTeX nötig sind, werden unter
Windows im MikTeX-Paket geliefert (siehe
\webaddress{http://www.mmi.rwth-aachen.de/Lehre/Seminare/Download}). Zur
Eingabe des Quellcodes kann prinzipiell jeder Editor verwendet werden, 
mit dem ASCII-Dateien geschrieben werden können. Empfehlenswert ist aber 
die Verwendung spezieller Entwicklungsumgebungen wie zum Beispiel TexnicCenter 
(siehe Downloads), die die beschriebenen Abläufe automatisieren. Viewer,
wie Yap (zur Anzeige von dvi-Formaten) oder GhostView (zur Anzeige von 
PostScript-Dateien) werden getrennt installiert, stehen dann allerdings
in die Entwicklungsumgebungen integriert zur Verfügung.

\subsubsection{Grafiken in LaTeX-Dokumenten}

Grundsätzlich sollen Grafiken für diese Ausarbetung als JPG/PNG/GIF eingefügt werden. Es vorteilhaft, die Bilder direkt auf die benötigte Größe zu bringen.
Maximale Breite ist die Textbreite von 13,7\,cm. Die maximale sinnvolle
Auflösung für Biler ist 150\,dpi. Höhere Auflösungen gehen beim Kopiervorgang
verloren. Die Bildgröße kann außerdem als Faktor zur textbreite angegeben werden.

% Aha, ein Trick! Ich will einen Befehl in Schreibmaschinenschrift schreiben, 
% aber es sind zuviele Sonderzeichen ( \ {} ) darin für \texttt.
% Also nehme ich den  \verb (wie verbatim) Befehl. Zusätzlich nutze ich,
% dass statt der geschweiften Klammern bei \verb im Prinzip ein beliebiges
% Zeichen, das kein Buchstabe ist verwendet werden kann. Dieses darf logischer
% Weise nicht innerhalb der "Klammer" vorkommen. Üblich sind + ! *  

Das Einscannen von Bildern sollte soweit wie möglich vermieden werden. Zum
einen ist die Qualität meist nicht gut und die Dateien sehr groß. Zum anderen
ist das Verwenden von Teilen aus Büchern selten gestattet. Optimal ist eine 
Nachahmung der Abbildung mit Verweis auf das Original: \verb+nach \cite{OriginalAutor}+. 
Sollte dies nicht gehen, kann aus Artikeln das Bild oft mit Grafikprogrammen 
extrahiert werden. Bei Scans sollte darauf geachtet werden, dass das Rauschen möglichst klein ist, 
\ie\ ein weisser Hintergrund sollte keine schwarzen Pünktchen und durchscheinende Seiten enthalten.

\subsection{Organisation der Dateien}

% Beispiel für eine Aufzählung

Um diesen Text texen zu können sind eine Reihe von Dateien nötig. Ihre
Aufgaben sind im Folgenden kurz beschrieben:

\begin{itemize}
\item \texttt{frame.tex}\\
  Dient als Ersatz für die Master-Datei des gesamten Seminarbandes. Der
  Befehl \verb+\selectlanguage{ngerman} %choose: ngerman or English
%%%%% Angaben für die Titelseite

% Titel der Arbeit
\Vtitle{Seminar MMI - Vorlage für \\Schriftliche Ausarbeitungen}
% Autor der Arbeit
\Vauthor{Ömer Emre Mutlu}
% e-mail des Autors
\Vaddress{emre.mutlu@rwth-aachen.de}
% Kurzfassung der Arbeit
\Vabstract{
Dieses Dokument dient als Anleitung und als Vorlage für die schriftlichen 
Ausarbeitungen für Seminare am \emph{Institut für Mensch-Maschine-Interaktion 
(MMI)}. Einsteiger in \LaTeX\ sollten parallel mit einem Ausdruck dieser
Vorlage und dem Quellcode arbeiten, um so leicht entspechende Befehle für
ein gewünschtes Layout zu finden. Helfen dieses Dokument sowie angegebene
Literatur nicht weiter, stehen natürlich die Betreuer für Fragen bereit. 
}
% Schlüsselworte zum Inhalt der Arbeit
\Vkeywords{\LaTeX, mmiSeminar-Style, Vorlage}
% tatsächliche Erstellung der Titelseite
\makeArticleTitle


% Beispiel für einen Abschnitt

\section{Einführung}
\label{sec:CSorg}

% Beispiel für Fußnoten
% Beispiel für Literaturangaben

\LaTeX\ basiert auf \TeX\footnote{sprich: Tech}, einem Textsatzsystem, das von 
Donald E. Knuth \cite{knuth} entwickelt und veröffentlicht wurde, um wissenschaftlichen 
Texten ein einheitliches und gutes Layout zu geben. Aufgrund seiner komplexen Syntax 
ist die direkte Verwendung von \TeX\ jedoch nicht leicht, deshalb hat Leslie Lamport 
\cite{latexManual} das Makro-Paket \LaTeX\ entwickelt. Dieses definiert Makros, 
die es dem Benutzer erlauben, komplexere und häufig benötigte Folgen von \TeX-Kommandos 
zu verwenden. Die neuste und vereinheitlichte Version ist \LaTeXe\, das wir hier verwenden.

% Beispiel für einen Unter-Abschnitt

\subsection{Warum \LaTeX}

% Beispiel für Hervorhebungen
% Beispiel für Sonderkommando \zb

\LaTeX\ (wegen besserer Lesbarkeit auch einfach LaTeX) ist eine Makro-Sprache 
und muss ähnlich einer Programmiersprache erlernt werden. Entsprechend muss 
TeX-Code erst compiliert oder "`\emph{getext}"' werden, bevor das tatsächliche 
Layout sichtbar wird. Doch die Vorteile der Verwendung von LaTeX für professionelle 
wissenschaftliche Texte treten besonders bei längeren (etwa 100 Seiten) Texten 
mit vielen Bildern und Formeln zutage. LaTeX setzt Grundregeln um, an denen sich auch 
professionelle Layouter orientieren, um einen Text gut lesbar und optisch ansprechend 
zu gestalten. Zusätzlich wird durch die festen Befehle (\zB\ für Kapitelanfänge) ein hohes Maß 
an Konsistenz erreicht. Arbeiten im technischen Bereich werden daher häufig in TeX ausgeführt.
Zum Beispiel verlangen viele Institute verlangen, dass Studien- und Diplomarbeiten in LaTeX geschrieben
sind. Fast alle internationalen Zeitschriften und Konferenzen erwarten TeX-Dokumente, 
für die sie entsprechende eigene \emph{Styles} entwickelt haben.

\subsection{Vom Quelltext zum PostScript}

\subsubsection{Dateien rund um \LaTeX}

% Beispiel für Schreibmaschinen-Schriftart

Der Quellcode für jedes LaTeX-Dokument steht in einer \emph{.tex}-Datei. 
Mit dem Befehl \texttt{input} können weitere Dateien eingebunden werden.
Dies hat den gleichen Effekt, als ob der Quellcode an Stelle des input-Befehls 
stände\footnote{siehe auch Datei \texttt{frame.tex}}. Befehle in LaTeX beginnen 
übrigens immer mit einem backslash $\backslash$.

In wissenschaftlichen Arbeiten werden sehr häufig Ergebnisse anderer Forscher 
zitiert. Im technischen Bereich geschieht dies durch eine Abkürzung oder 
Indexnummer, die im \emph{Literaturverzeichnis} wiederzufinden ist. 
Dort stehen dann genaue Angaben zu der jeweiligen Veröffentlichung. 
In LaTeX gibt es ein einheitliches Format in dem die Daten einer Veröffentlichung, 
wie \zB\ Autor, Titel, Jahr, usw.\ gespeichert werden. Mittels verschiedener 
Bibliography-Styles kann das Layout des Literaturverzeichnisses leicht verändert werden. 
Die Angaben zu den zitierten Arbeiten stehen in der Bibliography (\emph{.bib}-Datei),
die dann von BiBTeX ausgewertet wird.

Zur Umsetzung verschiedener Layouts im Text wird ein ähnliches Vorgehen wie
für das Literaturverzeichnis angewandt. Zunächst gibt es die \emph{Dokumentenklasse} 
(Befehl \texttt{documentclass} in frame.tex), die die grundsätzliche Formatierung festlegt. 
Dann erweitern die Styles (in \emph{.sty}-Dateien) die Dokumentenklassen 
und bieten weitere Makros an, die die Arbeit mit LaTeX vereinfachen sollen. 
Ein Beispiel ist iso8859-1.sty: Ohne diesen Style kann man Umlaute nicht direkt in den Text schreiben. 
Der mmiSeminar-Style dagegen dient der einheitlichen Formatierung und einfachen Erstellung der 
schriftlichen Ausarbeitungen zur späteren Zusammenführung in einem Seminarband.

\subsubsection{Von \LaTeX\ zu PostScript}

% Beispiel für Gedankenstriche

Der erste Schritt um aus dem geschriebenen Quellcode eine formatierte
PostScript-Version zu erstellen ist das so genannte "`\emph{Texen}"' --- das
Ausführen des \texttt{latex} Befehls. Dies geht entweder über die Konsole mit
Angabe des Dateinamens der Master-Datei (hier: \texttt{frame}), oder per
Menüpunkt im verwendeten Editor. In diesem Schritt werden einige Hilfsdateien
mit gleichem Namen wie die Master-Datei erstellt. Der eigentliche Text steht
in der \emph{.dvi}-Datei (dvi steht für device independent). 
Unter Linux und unter Windows existieren Programme zur Anzeige dieser Dateien.

Je nachdem, welche Änderungen im Quellcode vorgenommen wurden, muss \texttt{latex} bis
zu dreimal ausgeführt werden. Um alle Verweise zu finden, erstellt das Programm
zunächst Tabellen, die im nächsten Schritt verwendet werden. Beim ersten Texen
wird auch eine Liste der Literaturverweise (\texttt{cite}) erstellt. Wurden
die .bib-Datei verändert muss auch der Befehl \texttt{bibtex Master-Datei}
ausgeführt werden. Hierauf folgen in der Regel zwei Aufrufe von \texttt{latex}, 
bis alle Referenzen korrekt sind. Komfortable LaTeX-Entwicklungsumgebungen unter Linux 
und unter Windows führen die notwendigen Schritte auf Knopfdruck automatisch aus.

Aus der nun korrekten .dvi-Datei wird eine PostScript-Datei (.ps) mit dem
Befehl \texttt{dvips -t a4 Master-File} erzeugt. Der Parameter \texttt{-t}
gibt die Papiergröße A4 an. Die Endung des Master-Files ist wieder
wegzulassen. Das Ergebnis kann unter Linux mit \texttt{gv} oder unter 
Windows mit \texttt{ghostview} betrachtet werden.

% Die tabular-Umgebung eignet sich nicht nur für Tabellen, sondern auch zum Anordnen anderer Textteile

Für das Seminar ist die Befehlsfolge, die bei korrektem Quellcode immer zu
einem PostScript führen sollte:
{\tt
  \begin{center}
    \begin{tabular}[c]{ll}
      latex & frame \\
      bibtex & frame \\
      latex & frame \\
      latex & frame \\
    \end{tabular}
  \end{center}
}

\subsection{Tools}
\label{sub:CStools}

Die für Linux und Windows zur Verfügung stehenden Hilfsmittel unterscheiden
sich, obwohl viele der Linux-üblichen Programme ebenso für Windows erhältlich sind. 
LaTeX kommt aus der Unix-Welt und daher ist hier etwas leichter zu verwenden, 
insbesondere sind die benötigten Programme bereits installiert. 
Aber auch unter Windows sind alle Anwendungen, die zum Arbeiten mit LaTeX
gebraucht werden, kostenlos verfügbar. Die jeweiligen Werkzeuge sind in 
den nächsten Abschnitten kurz beschrieben.

\subsubsection{Unter Linux}

Die Programmepakete LaTeX, Emacs und dvips müssen installiert werden.
Emacs bietet für praktisch alle unter Linux üblichen Dateiformate Syntax-Highlighting 
und besondere Makros, die das Programmieren vereinfachen.
Dies trifft auch für .tex und .bib-Dateien zu. Die Kombination \texttt{Ctrl-C
  Ctrl-E} erzeugt \zB\ beliebige LaTeX-Umgebungen. Für BiBTeX erscheint ein
Menüpunkt, mit dem sich neue Einträge für das Literaturverzeichnis generieren
lassen. PostScript-Dateien können mit den kostenlos in der Distribution
enthaltenen Programmen ghostview und gv geöffnet werden.
Grafiken können unter Linux mit xfig erstellt werden. 

% Beispiel für Web-Adresse

\subsubsection{Unter Windows}

Alle Routinen, die für die Verwendung von LaTeX nötig sind, werden unter
Windows im MikTeX-Paket geliefert (siehe
\webaddress{http://www.mmi.rwth-aachen.de/Lehre/Seminare/Download}). Zur
Eingabe des Quellcodes kann prinzipiell jeder Editor verwendet werden, 
mit dem ASCII-Dateien geschrieben werden können. Empfehlenswert ist aber 
die Verwendung spezieller Entwicklungsumgebungen wie zum Beispiel TexnicCenter 
(siehe Downloads), die die beschriebenen Abläufe automatisieren. Viewer,
wie Yap (zur Anzeige von dvi-Formaten) oder GhostView (zur Anzeige von 
PostScript-Dateien) werden getrennt installiert, stehen dann allerdings
in die Entwicklungsumgebungen integriert zur Verfügung.

\subsubsection{Grafiken in LaTeX-Dokumenten}

Grundsätzlich sollen Grafiken für diese Ausarbetung als JPG/PNG/GIF eingefügt werden. Es vorteilhaft, die Bilder direkt auf die benötigte Größe zu bringen.
Maximale Breite ist die Textbreite von 13,7\,cm. Die maximale sinnvolle
Auflösung für Biler ist 150\,dpi. Höhere Auflösungen gehen beim Kopiervorgang
verloren. Die Bildgröße kann außerdem als Faktor zur textbreite angegeben werden.

% Aha, ein Trick! Ich will einen Befehl in Schreibmaschinenschrift schreiben, 
% aber es sind zuviele Sonderzeichen ( \ {} ) darin für \texttt.
% Also nehme ich den  \verb (wie verbatim) Befehl. Zusätzlich nutze ich,
% dass statt der geschweiften Klammern bei \verb im Prinzip ein beliebiges
% Zeichen, das kein Buchstabe ist verwendet werden kann. Dieses darf logischer
% Weise nicht innerhalb der "Klammer" vorkommen. Üblich sind + ! *  

Das Einscannen von Bildern sollte soweit wie möglich vermieden werden. Zum
einen ist die Qualität meist nicht gut und die Dateien sehr groß. Zum anderen
ist das Verwenden von Teilen aus Büchern selten gestattet. Optimal ist eine 
Nachahmung der Abbildung mit Verweis auf das Original: \verb+nach \cite{OriginalAutor}+. 
Sollte dies nicht gehen, kann aus Artikeln das Bild oft mit Grafikprogrammen 
extrahiert werden. Bei Scans sollte darauf geachtet werden, dass das Rauschen möglichst klein ist, 
\ie\ ein weisser Hintergrund sollte keine schwarzen Pünktchen und durchscheinende Seiten enthalten.

\subsection{Organisation der Dateien}

% Beispiel für eine Aufzählung

Um diesen Text texen zu können sind eine Reihe von Dateien nötig. Ihre
Aufgaben sind im Folgenden kurz beschrieben:

\begin{itemize}
\item \texttt{frame.tex}\\
  Dient als Ersatz für die Master-Datei des gesamten Seminarbandes. Der
  Befehl \verb+\selectlanguage{ngerman} %choose: ngerman or English
%%%%% Angaben für die Titelseite

% Titel der Arbeit
\Vtitle{Seminar MMI - Vorlage für \\Schriftliche Ausarbeitungen}
% Autor der Arbeit
\Vauthor{Ömer Emre Mutlu}
% e-mail des Autors
\Vaddress{emre.mutlu@rwth-aachen.de}
% Kurzfassung der Arbeit
\Vabstract{
Dieses Dokument dient als Anleitung und als Vorlage für die schriftlichen 
Ausarbeitungen für Seminare am \emph{Institut für Mensch-Maschine-Interaktion 
(MMI)}. Einsteiger in \LaTeX\ sollten parallel mit einem Ausdruck dieser
Vorlage und dem Quellcode arbeiten, um so leicht entspechende Befehle für
ein gewünschtes Layout zu finden. Helfen dieses Dokument sowie angegebene
Literatur nicht weiter, stehen natürlich die Betreuer für Fragen bereit. 
}
% Schlüsselworte zum Inhalt der Arbeit
\Vkeywords{\LaTeX, mmiSeminar-Style, Vorlage}
% tatsächliche Erstellung der Titelseite
\makeArticleTitle


% Beispiel für einen Abschnitt

\section{Einführung}
\label{sec:CSorg}

% Beispiel für Fußnoten
% Beispiel für Literaturangaben

\LaTeX\ basiert auf \TeX\footnote{sprich: Tech}, einem Textsatzsystem, das von 
Donald E. Knuth \cite{knuth} entwickelt und veröffentlicht wurde, um wissenschaftlichen 
Texten ein einheitliches und gutes Layout zu geben. Aufgrund seiner komplexen Syntax 
ist die direkte Verwendung von \TeX\ jedoch nicht leicht, deshalb hat Leslie Lamport 
\cite{latexManual} das Makro-Paket \LaTeX\ entwickelt. Dieses definiert Makros, 
die es dem Benutzer erlauben, komplexere und häufig benötigte Folgen von \TeX-Kommandos 
zu verwenden. Die neuste und vereinheitlichte Version ist \LaTeXe\, das wir hier verwenden.

% Beispiel für einen Unter-Abschnitt

\subsection{Warum \LaTeX}

% Beispiel für Hervorhebungen
% Beispiel für Sonderkommando \zb

\LaTeX\ (wegen besserer Lesbarkeit auch einfach LaTeX) ist eine Makro-Sprache 
und muss ähnlich einer Programmiersprache erlernt werden. Entsprechend muss 
TeX-Code erst compiliert oder "`\emph{getext}"' werden, bevor das tatsächliche 
Layout sichtbar wird. Doch die Vorteile der Verwendung von LaTeX für professionelle 
wissenschaftliche Texte treten besonders bei längeren (etwa 100 Seiten) Texten 
mit vielen Bildern und Formeln zutage. LaTeX setzt Grundregeln um, an denen sich auch 
professionelle Layouter orientieren, um einen Text gut lesbar und optisch ansprechend 
zu gestalten. Zusätzlich wird durch die festen Befehle (\zB\ für Kapitelanfänge) ein hohes Maß 
an Konsistenz erreicht. Arbeiten im technischen Bereich werden daher häufig in TeX ausgeführt.
Zum Beispiel verlangen viele Institute verlangen, dass Studien- und Diplomarbeiten in LaTeX geschrieben
sind. Fast alle internationalen Zeitschriften und Konferenzen erwarten TeX-Dokumente, 
für die sie entsprechende eigene \emph{Styles} entwickelt haben.

\subsection{Vom Quelltext zum PostScript}

\subsubsection{Dateien rund um \LaTeX}

% Beispiel für Schreibmaschinen-Schriftart

Der Quellcode für jedes LaTeX-Dokument steht in einer \emph{.tex}-Datei. 
Mit dem Befehl \texttt{input} können weitere Dateien eingebunden werden.
Dies hat den gleichen Effekt, als ob der Quellcode an Stelle des input-Befehls 
stände\footnote{siehe auch Datei \texttt{frame.tex}}. Befehle in LaTeX beginnen 
übrigens immer mit einem backslash $\backslash$.

In wissenschaftlichen Arbeiten werden sehr häufig Ergebnisse anderer Forscher 
zitiert. Im technischen Bereich geschieht dies durch eine Abkürzung oder 
Indexnummer, die im \emph{Literaturverzeichnis} wiederzufinden ist. 
Dort stehen dann genaue Angaben zu der jeweiligen Veröffentlichung. 
In LaTeX gibt es ein einheitliches Format in dem die Daten einer Veröffentlichung, 
wie \zB\ Autor, Titel, Jahr, usw.\ gespeichert werden. Mittels verschiedener 
Bibliography-Styles kann das Layout des Literaturverzeichnisses leicht verändert werden. 
Die Angaben zu den zitierten Arbeiten stehen in der Bibliography (\emph{.bib}-Datei),
die dann von BiBTeX ausgewertet wird.

Zur Umsetzung verschiedener Layouts im Text wird ein ähnliches Vorgehen wie
für das Literaturverzeichnis angewandt. Zunächst gibt es die \emph{Dokumentenklasse} 
(Befehl \texttt{documentclass} in frame.tex), die die grundsätzliche Formatierung festlegt. 
Dann erweitern die Styles (in \emph{.sty}-Dateien) die Dokumentenklassen 
und bieten weitere Makros an, die die Arbeit mit LaTeX vereinfachen sollen. 
Ein Beispiel ist iso8859-1.sty: Ohne diesen Style kann man Umlaute nicht direkt in den Text schreiben. 
Der mmiSeminar-Style dagegen dient der einheitlichen Formatierung und einfachen Erstellung der 
schriftlichen Ausarbeitungen zur späteren Zusammenführung in einem Seminarband.

\subsubsection{Von \LaTeX\ zu PostScript}

% Beispiel für Gedankenstriche

Der erste Schritt um aus dem geschriebenen Quellcode eine formatierte
PostScript-Version zu erstellen ist das so genannte "`\emph{Texen}"' --- das
Ausführen des \texttt{latex} Befehls. Dies geht entweder über die Konsole mit
Angabe des Dateinamens der Master-Datei (hier: \texttt{frame}), oder per
Menüpunkt im verwendeten Editor. In diesem Schritt werden einige Hilfsdateien
mit gleichem Namen wie die Master-Datei erstellt. Der eigentliche Text steht
in der \emph{.dvi}-Datei (dvi steht für device independent). 
Unter Linux und unter Windows existieren Programme zur Anzeige dieser Dateien.

Je nachdem, welche Änderungen im Quellcode vorgenommen wurden, muss \texttt{latex} bis
zu dreimal ausgeführt werden. Um alle Verweise zu finden, erstellt das Programm
zunächst Tabellen, die im nächsten Schritt verwendet werden. Beim ersten Texen
wird auch eine Liste der Literaturverweise (\texttt{cite}) erstellt. Wurden
die .bib-Datei verändert muss auch der Befehl \texttt{bibtex Master-Datei}
ausgeführt werden. Hierauf folgen in der Regel zwei Aufrufe von \texttt{latex}, 
bis alle Referenzen korrekt sind. Komfortable LaTeX-Entwicklungsumgebungen unter Linux 
und unter Windows führen die notwendigen Schritte auf Knopfdruck automatisch aus.

Aus der nun korrekten .dvi-Datei wird eine PostScript-Datei (.ps) mit dem
Befehl \texttt{dvips -t a4 Master-File} erzeugt. Der Parameter \texttt{-t}
gibt die Papiergröße A4 an. Die Endung des Master-Files ist wieder
wegzulassen. Das Ergebnis kann unter Linux mit \texttt{gv} oder unter 
Windows mit \texttt{ghostview} betrachtet werden.

% Die tabular-Umgebung eignet sich nicht nur für Tabellen, sondern auch zum Anordnen anderer Textteile

Für das Seminar ist die Befehlsfolge, die bei korrektem Quellcode immer zu
einem PostScript führen sollte:
{\tt
  \begin{center}
    \begin{tabular}[c]{ll}
      latex & frame \\
      bibtex & frame \\
      latex & frame \\
      latex & frame \\
    \end{tabular}
  \end{center}
}

\subsection{Tools}
\label{sub:CStools}

Die für Linux und Windows zur Verfügung stehenden Hilfsmittel unterscheiden
sich, obwohl viele der Linux-üblichen Programme ebenso für Windows erhältlich sind. 
LaTeX kommt aus der Unix-Welt und daher ist hier etwas leichter zu verwenden, 
insbesondere sind die benötigten Programme bereits installiert. 
Aber auch unter Windows sind alle Anwendungen, die zum Arbeiten mit LaTeX
gebraucht werden, kostenlos verfügbar. Die jeweiligen Werkzeuge sind in 
den nächsten Abschnitten kurz beschrieben.

\subsubsection{Unter Linux}

Die Programmepakete LaTeX, Emacs und dvips müssen installiert werden.
Emacs bietet für praktisch alle unter Linux üblichen Dateiformate Syntax-Highlighting 
und besondere Makros, die das Programmieren vereinfachen.
Dies trifft auch für .tex und .bib-Dateien zu. Die Kombination \texttt{Ctrl-C
  Ctrl-E} erzeugt \zB\ beliebige LaTeX-Umgebungen. Für BiBTeX erscheint ein
Menüpunkt, mit dem sich neue Einträge für das Literaturverzeichnis generieren
lassen. PostScript-Dateien können mit den kostenlos in der Distribution
enthaltenen Programmen ghostview und gv geöffnet werden.
Grafiken können unter Linux mit xfig erstellt werden. 

% Beispiel für Web-Adresse

\subsubsection{Unter Windows}

Alle Routinen, die für die Verwendung von LaTeX nötig sind, werden unter
Windows im MikTeX-Paket geliefert (siehe
\webaddress{http://www.mmi.rwth-aachen.de/Lehre/Seminare/Download}). Zur
Eingabe des Quellcodes kann prinzipiell jeder Editor verwendet werden, 
mit dem ASCII-Dateien geschrieben werden können. Empfehlenswert ist aber 
die Verwendung spezieller Entwicklungsumgebungen wie zum Beispiel TexnicCenter 
(siehe Downloads), die die beschriebenen Abläufe automatisieren. Viewer,
wie Yap (zur Anzeige von dvi-Formaten) oder GhostView (zur Anzeige von 
PostScript-Dateien) werden getrennt installiert, stehen dann allerdings
in die Entwicklungsumgebungen integriert zur Verfügung.

\subsubsection{Grafiken in LaTeX-Dokumenten}

Grundsätzlich sollen Grafiken für diese Ausarbetung als JPG/PNG/GIF eingefügt werden. Es vorteilhaft, die Bilder direkt auf die benötigte Größe zu bringen.
Maximale Breite ist die Textbreite von 13,7\,cm. Die maximale sinnvolle
Auflösung für Biler ist 150\,dpi. Höhere Auflösungen gehen beim Kopiervorgang
verloren. Die Bildgröße kann außerdem als Faktor zur textbreite angegeben werden.

% Aha, ein Trick! Ich will einen Befehl in Schreibmaschinenschrift schreiben, 
% aber es sind zuviele Sonderzeichen ( \ {} ) darin für \texttt.
% Also nehme ich den  \verb (wie verbatim) Befehl. Zusätzlich nutze ich,
% dass statt der geschweiften Klammern bei \verb im Prinzip ein beliebiges
% Zeichen, das kein Buchstabe ist verwendet werden kann. Dieses darf logischer
% Weise nicht innerhalb der "Klammer" vorkommen. Üblich sind + ! *  

Das Einscannen von Bildern sollte soweit wie möglich vermieden werden. Zum
einen ist die Qualität meist nicht gut und die Dateien sehr groß. Zum anderen
ist das Verwenden von Teilen aus Büchern selten gestattet. Optimal ist eine 
Nachahmung der Abbildung mit Verweis auf das Original: \verb+nach \cite{OriginalAutor}+. 
Sollte dies nicht gehen, kann aus Artikeln das Bild oft mit Grafikprogrammen 
extrahiert werden. Bei Scans sollte darauf geachtet werden, dass das Rauschen möglichst klein ist, 
\ie\ ein weisser Hintergrund sollte keine schwarzen Pünktchen und durchscheinende Seiten enthalten.

\subsection{Organisation der Dateien}

% Beispiel für eine Aufzählung

Um diesen Text texen zu können sind eine Reihe von Dateien nötig. Ihre
Aufgaben sind im Folgenden kurz beschrieben:

\begin{itemize}
\item \texttt{frame.tex}\\
  Dient als Ersatz für die Master-Datei des gesamten Seminarbandes. Der
  Befehl \verb+\input{MyText}+ fügt dieses Dokument (MyText.tex) ein. \glqq
  MyText\grqq\ ist durch den Namen der eigenen Datei zu ersetzen. Weiter unter
  wird mit dem Befehl \verb+\bibliography{MyBib}+ festgelegt, dass die
  Literaturhinweise in der Datei \glqq MyBib.bib\grqq\ stehen. Diese sollte
  bei den Seminararbeiten den gleichen Namen wie der Text haben.
  
\item \texttt{MyText.tex}\\
  Der Quelltext dieser Anleitung. Neben dieser Funktion dient die Datei auch
  als Vorlage für die Seminararbeiten. Mit Hilfe der PostScript Version und
  des Quellcodes können die Befehle um ein bestimmtes Layout zu erhalten
  leicht gefunden werden. Die eigene Arbeit soll den Namen
  \texttt{nachname.tex} haben (alles klein).
  
\item \texttt{MyBib.bib}\\
  Enthält die Literaturangaben zu diesem Text. Die Datei enthält weiterhin
  Vorlagen für andere Typen von Literaturangaben für die Nicht-Emacs-User. Das
  eigene Literaturverzeichnis soll den Namen \texttt{nachname.bib} haben
  (alles klein).
  
\item \texttt{xxx.tex xxx.sty xxx.bst}\\
  In diesen Dateien werden Styles und Makros definiert, die Erweiterungen oder
  Veränderungen von LaTeX darstellen. Sie dürfen auf keinen Fall verändert
  werden. 
  
\end{itemize}

Weiterhin ist sehr wichtig, dass alle Bilder als .jpg, .png oder .gif gespeichert sind und
sich im Unterverzeichnis \texttt{bilder} relativ zu eurem Hauptfile befinden. 
Wenn also euer \texttt{nachname.tex} in \texttt{home/username/seminar steht}, gehören
die Bilder in \texttt{home/username/seminar/bilder}. 

\section{Sehr kurze Einführung in \LaTeX}

\label{sec:CSlatex}

Für komplexere Aufgaben sind weitere Nachschlagewerke unverzichtbar. 
Eine recht umfassende Beschreibung ist kostenlos erhältlich:
\glqq\emph{The Not So Short Introduction to \LaTeX2e}\grqq \cite{oetiker},
eine Einführung in Latex. Für schwierige Formeln ist das \AmS-\LaTeX-Package
sehr zu empfehlen. Das \glqq\emph{\AmS-\LaTeX\ User's Guide} \cite{ams}
beschreibt diese häufig verwendete Erweiterung zu LaTex. Die beiden Dokumente
sind über die Seminar-Seite
(\webaddress{http://www.mmi.rwth-aachen.de/Lehre/Seminare/Download})
verlinkt.  Weiterhin gibt es einige durchaus empfehlenswerte Bücher: 
\glqq\emph{\LaTeX: A Document Preparation System}\grqq, das Reference Manual \cite{latexManual} 
und \glqq\emph{The \LaTeX\ Companion}\grqq \cite{latexCompanion}, Erklärungen zu verschiedenen
Zusatzpackages u.\,a.\ \AmS-\LaTeX. Ausserdem gibt es noch die drei Latex
Bücher von Kopka, von denen nur das erste \glqq\emph{\LaTeX\ Einführung}\grqq
\cite{kopka1} für den einfachen Anwender von LaTex nötig ist.

\subsection{Überschriften und Referenzen}

\label{sub:CSheadlines}

Wie in diesem Beispieldokument zu sehen ist, gibt es für Artikel drei
Hierarchieebenen der Überschriften. Sie werden mit den Befehlen
\verb+\section{...}+, \verb+\subsection{...}+ und \verb+\subsubsection{...}+
gesetzt. Dabei ist \texttt{...} durch die jeweilig Überschrift zu
ersetzen. Innerhalb jedes Blocks kann durch den Befehl \texttt{label} eine
Marke für spätere Referenzierung gesetzt werden. In allen sochen Marken sollten
wiederum (wie bei Bilddateien) am Anfang stehen. Für diesen Abschnitt ist im
Quellcode der Befehl \verb+\label{sec:CSlatex}+ zu finden. Mit dem Befehl
\verb+\ref{sec:CSlatex}+ wird die Abschnittsnummer dieses Abschnitts
ausgegeben: 2. Für jede Überschriftenebene, sowie Formeln, Aufzählungspunkte,
Definitionen usw.\ gibt es eigene Zähler, die entsprechen mit label gesetzt
und mit ref wieder abgerufen werden können (s. Quelltext). Es zählt jeweils
die unterste Stufe. Daher ist es sinnvoll, den label Befehl direkt nach einem
einen neuen Block einleitenden Befehl zu verwenden.

Absätze werden in LaTeX durch eine oder mehrere Leerzeilen erreicht. Alle
überflüssigen Leerzeichen und -zeilen werden ignoriert. Befehle wie \verb+\\+
(newline), \verb+\clearpage+ oder \verb+\newpage+ sind nur in Sonderfällen zu
verwenden und in dieser Arbeit gar nicht.

\subsection{Textformatierung}
\label{sub:CSformat}

LaTeX bietet einige Möglichkeiten, die Schriftart zu verändern. Diese sollten
sparsam eingesetzt werden, um die Lesbarkeit des Textes nicht zu
beeinträchtigen. \emph{Wichtige} oder \emph{neue} Begriffe werden betont
(engl. emphasize) mit dem Befehl \verb+\emph{...}+. Die
Schreibmaschinenenschrift \verb+\texttt{...}+ wird für Befehle o\,ä.\ 
verwendet. Bei Befehlen, die LaTeX-Steuerzeichen enthalten empfiehlt sich der
Befehl \verb \verb+...+ \ (vgl. Kommentar im Quellcode in Abschnitt
\ref{sub:CStools}).

\subsection{Aufzählungen}
\label{sub:CSlists}

LaTeX stellt im wesentlichen zwei Arten von Aufzählungen zur Verfügung:
nummerierte und unnummerierte. Es handelt sich dabei um so genannte Umgebungen
(engl. environments), die immer von \verb+\begin{env}+ und \verb+\end{env}+
umschlossen werden. Dabei ist \texttt{env} durch den Namen der jeweiligen
Umgebung zu ersetzen. Wer Emacs verwendet kann mit \texttt{Ctrl-c Ctrl-e}
leicht Environments erstellen. Eine spezifische Aufzählungsumgebung für
Algorithmen wird in Abschnitt \ref{sub:CSmmiLists} erläutert. Es folgen
zwei Beispiele zu den beiden Aufzählungsarten.

\subsubsection{Itemize-Umgebung}
Dient der nicht nummerierten Aufzählung. Beispiel:
\begin{itemize}
\item Item1
\item Item2
\item Item3
\end{itemize}

\subsubsection{Enumerate-Umgebung}
Dient der nummerierten Aufzählung. Beispiel:
\begin{enumerate}
\item Item1
\item Item2
\item Item3 \label{enum:CSsubenum}
  \begin{enumerate}
  \item Item31
  \item Item32
  \item Item33
  \end{enumerate}
\end{enumerate}
Der \texttt{label}-Befehl funktioniert auch innerhalb von Aufzählungen 
(siehe Quellcode): Punkt \ref{enum:CSsubenum} ist in mehrere Unterpunkte aufgeteilt.

\subsection{Formel-Darstellung}
\label{sub:CSmath}

LaTex bietet zahlreiche komfortable Möglichkeiten, um Formeln darzustellen. 
Da aber die Darstellung sehr komplexer Formeln ermöglicht wird, existieren entsprechend viele 
Befehle zu ihrer Beschreibung. Zunächst muss man zwischen mathematischen
Ausdrücken im Text, wie $\alpha\in\mathbb{R}$, und abgesetzten Formeln oder
Ausdrücken entscheiden. Es folgen einige Beispiel aus anderen Texten, die
unter Zuhilfenahme des Quellcodes den Umgang mit Formeln verdeutlichen
sollten.

Hier zunächst zwei einfache Formeln.

\begin{equation}
  h_i(z)=\frac{1}{\sqrt{2\pi}\sigma_i}\cdot e^{-\left(\frac{z^2}{2\sigma_i^2}\right)}
\end{equation}
\begin{equation}
o_j=f_{\mathrm{sig}} \left( \sum_{i=1}^N c_{ij} h(||x-w_i||) \right)
\end{equation}

Mit \verb+\intertext{...}+ lässt sich Text innerhalb ausgerichteter
Formelblöcke einfügen.

\begin{align}
    w_i(t+1) &= w_i(t)-\alpha(t)\left[x(t)-w_i(t)\right]\nonumber \\ 
    w_j(t+1) &= w_j(t)+\alpha(t)\left[x(t)-w_j(t)\right]
  \label{eq:CSlvq2}
  \intertext{where $w_i$ must be of a different class than $x$ and $w_j$ of
    the same class. If the two closest prototypes are of the same class
    (training of type (2) the following modification scheme is applicable (no
    "window"\ constraint):}
  w_k(t+1) &= w_k(t) + \epsilon\alpha(t)\left[ x(t) - w_k(t)\right]
  \label{eq:CSlvq3}
\end{align}

\begin{equation}
\begin{split}
\label{eq:CScov}
\Sigma &= E\{(\mathbf{\mathbf{x}}-\mathbf{m})(\mathbf{\mathbf{x}}-\mathbf{m})^T\} \\
&= \left[ \begin{array}{ccc}
    E\{(x_1-m_1)(x_1-m_1)\} & \cdots & E\{(x_1-m_1)(x_n-m_n)\}\\
    \vdots & & \vdots \\
    E\{(x_n-m_n)(x_1-m_1)\} & \cdots & E\{(x_n-m_n)(x_n-m_n)\}\\
       \end{array} \right] \\
     &= \left[ \begin{array}{ccc}
         c_{11} & \cdots & c_{1n} \\
         \vdots & & \vdots \\
         c_{n1} & \cdots & c_{nn}
       \end{array} \right]
\end{split}
\end{equation}
Auch Formeln können natürlich Label besitzen, \zB\: Die Modifikation der Gewichte
beim LVQ3 Training \eqref{eq:CSlvq3} versteht ohne Erklärung keiner.


\section{Der mmiSeminar-Style}
\label{sec:CSstyle}

Der zunächst von Peter Dörfler für die Seminare am Lehrstuhl für Technische Informatik 
erstellte Style wird mit dem Wintersemester 06/07 auch für die Seminare des Instituts 
für Mensch-Maschine-Interaktion eingesetzt. Der Style bietet einige Makros, die das 
Erstellen von LaTeX Dokumenten vereinfachen. Durch Ändern einiger Verhaltensweisen 
von LaTeX wird es möglich, den Seminarband als ein TeX-Dokument zu behandeln. 
Weiterhin sind darin einheitliche Formatierungen und das Layout für den Text definiert.

\subsection{Die Titelseite}

\label{sub:CStitle}

Die Titelseite lässt sich recht schnell erstellen (siehe Quellcode). Es müssen
nur die folgenden Werte (eigentlich Makros) gesetzt werden:

\begin{narrowitems}
\item Der Autor des Textes
\item Der Titel
\item Die e-mail des Autors
\item Zusammenfassung
\item Schlüsselworte für den Inhalt
\end{narrowitems}
Der Befehl \verb+\makeArticleTitle+ erstellt dann die Titelseite. 

\subsection{Listen-Umgebungen im MMI-Style}
\label{sub:CSmmiLists}

\subsubsection{Aufzählungen}
Zwei Arten von Aufzählungen sind hinzugekommen:
\begin{narrowenum}
  \item \texttt{narrowenum}
  \item \texttt{narrowitems}
\end{narrowenum}
Die erste der beiden hat direkt hier Anwendung gefunden. Der andere
Aufzählungstyp ist in Abschnitt \ref{sub:CStitle} zu sehen. Diese Umgebungen
sind für Fälle gedacht, in denen die einzelnen Punkte sehr kurz sind. Die
normalen Abstände wirken einfach zu groß, wenn pro Punkt nur zwei, drei
Worte stehen.

\subsubsection{Algorithmus-Umgebung}

Die \texttt{algorithm}-Umgebung dient dazu Algorithmen
darzustellen. Sie funktioniert im Prinzip wie andere Listenumgebungen,
hat aber ein etwas anderes Layout. Zusätzlich sind alle weiteren Stufen nach
der ersten in die \texttt{subalgorithm}-Umgebung zu setzen. Üblicherweise wird
ein Algorithmus in eine \emph{floating}-Umgebung gesetzt. Das heisst, dass er
nicht unbedingt an der Stelle erscheint, an der der Algorithmus im Text steht, 
sondern eventuell in den nächsten freien Platz geschoben wird.

\begin{figure}[htb]
  \begin{algorithm}
  \item begin begin begin begin begin begin begin begin begin begin begin 
  			begin begin begin begin begin begin begin 
  \item \begin{subalgorithm}
    \item Item21 Item21 Item21 Item21 Item21 Item21 Item21 
    \item Item22 Item22 Item22 Item22 Item22 Item22 Item22 
    \end{subalgorithm}
  \item end
  \end{algorithm}
  \caption{Ein Beispiel-Algorithmus}
  \label{fig:CSalgo}
\end{figure}
Die \texttt{figure}-Umgebung (siehe Quellcode) ist für floating Bilder, mit der
Position als Parameter (\textbf{h}ere, \textbf{t}op, \textbf{b}ottom). Der
Befehl \texttt{caption} legt die Bildunterschrift fest. Diese Umgebung gehört
zu Standard-Latex.

\subsection{Bilder und Tabellen}
\label{sub:CSfigtab}

Sowohl für Tabellen als auch für Abbildungen gibt es im mmiSeminar-Style
floating-Umgebungen, die einfacher in der Anwendung sind als die
Standardumgebungen. 

\subsubsection{Abbildungen}

Abbildungen sollten im .jpg, .png oder .gif Format vorliegen und in der  \emph{figure}-Umgebung per \emph{includegraphics}-Befehl eingebunden werden. Die Bildgröße kann dabei auch relativ zur Textbreite angegeben werden, wie in Bild \ref{fig:Testbild} gezeigt.

\begin{figure}[htb]
 \centering \includegraphics[width=0.5\textwidth]{bilder/TestBild.png}
 \caption{Bildunterschrift für das Testbild. Sollte das Bild von einer anderen Quelle stammen wird hier auch mit cite darauf verwiesen.}
 \label{fig:Testbild}
\end{figure}

Auch in Abbildungen sollte es natürlich möglich sein, allen Text ohne Probleme
zu lesen. Leider sieht man oft das Gegenteil.

\subsubsection{Tabellen}

Die Tabellen-Umgebung \texttt{vtable} ist so ähnlich aufbebaut wie die
\texttt{xxxpsfig} Befehle weiter oben. Die Parameter sind in Reihenfolge:
\begin{narrowenum}
  \item Position
  \item Label
  \item Caption
  \item eigentliche Tabelle
\end{narrowenum}
Für die Labels gilt das gleich wie schon bei den Abbildungen. Der
Referenzierungsbefehl ist \texttt{tref}.  Die eigentliche Tabelle ist am
besten an einem Beispiel, Tabelle \tref{figs}, erklärt. Direkt nach dem Beginn
der \texttt{tabular}-Umgebung folgt ein Format-String, der die Ausrichtung des
Textes in den einzelnen Spalten festlegt. Die Optionen \texttt{l}, \texttt{r}
und \texttt{c} stehen für links, rechts und mittig ausgerichtet. Mit
\texttt{p\{Länge\}} kann man eine Spalte fester Breite deklarieren. Der
senkrechte Strich erzeugt einen ebensolchen. Vom Layout gefälliger und auch
lesbarer ist die Verwendung nur nötiger Linien. Die im Beispiel wäre \zB\ auch
nicht nötig gewesen. Mit \texttt{multiline\{spalten\}\{ausrichtung\}\{text\}}
kann man einen Text über mehrere Spalten ziehen und ausrichten wie angegeben.
Alles weitere steht in der angegebenen Literatur.

\appendixOfArticle
\section{Testing the Appendix}

Der Anhang wird durch den Befehl \texttt{appendixOfArticle} eingeleitet. Auf
keinen Fall ist der Befehl \texttt{appendix} zu verwenden. 

Im Anhang stehen solche Dinge, die nicht in den sonstigen Text passen. Lange
Beweise und zusätzliche Beispiele oder Abbildungen gehören dazu. Im
Allgemeinen sollte ein Anhang möglichst ganz vermieden werden.

%%% Local Variables: 
%%% mode: latex
%%% TeX-master: "frame"
%%% End: 







+ fügt dieses Dokument (MyText.tex) ein. \glqq
  MyText\grqq\ ist durch den Namen der eigenen Datei zu ersetzen. Weiter unter
  wird mit dem Befehl \verb+\bibliography{MyBib}+ festgelegt, dass die
  Literaturhinweise in der Datei \glqq MyBib.bib\grqq\ stehen. Diese sollte
  bei den Seminararbeiten den gleichen Namen wie der Text haben.
  
\item \texttt{MyText.tex}\\
  Der Quelltext dieser Anleitung. Neben dieser Funktion dient die Datei auch
  als Vorlage für die Seminararbeiten. Mit Hilfe der PostScript Version und
  des Quellcodes können die Befehle um ein bestimmtes Layout zu erhalten
  leicht gefunden werden. Die eigene Arbeit soll den Namen
  \texttt{nachname.tex} haben (alles klein).
  
\item \texttt{MyBib.bib}\\
  Enthält die Literaturangaben zu diesem Text. Die Datei enthält weiterhin
  Vorlagen für andere Typen von Literaturangaben für die Nicht-Emacs-User. Das
  eigene Literaturverzeichnis soll den Namen \texttt{nachname.bib} haben
  (alles klein).
  
\item \texttt{xxx.tex xxx.sty xxx.bst}\\
  In diesen Dateien werden Styles und Makros definiert, die Erweiterungen oder
  Veränderungen von LaTeX darstellen. Sie dürfen auf keinen Fall verändert
  werden. 
  
\end{itemize}

Weiterhin ist sehr wichtig, dass alle Bilder als .jpg, .png oder .gif gespeichert sind und
sich im Unterverzeichnis \texttt{bilder} relativ zu eurem Hauptfile befinden. 
Wenn also euer \texttt{nachname.tex} in \texttt{home/username/seminar steht}, gehören
die Bilder in \texttt{home/username/seminar/bilder}. 

\section{Sehr kurze Einführung in \LaTeX}

\label{sec:CSlatex}

Für komplexere Aufgaben sind weitere Nachschlagewerke unverzichtbar. 
Eine recht umfassende Beschreibung ist kostenlos erhältlich:
\glqq\emph{The Not So Short Introduction to \LaTeX2e}\grqq \cite{oetiker},
eine Einführung in Latex. Für schwierige Formeln ist das \AmS-\LaTeX-Package
sehr zu empfehlen. Das \glqq\emph{\AmS-\LaTeX\ User's Guide} \cite{ams}
beschreibt diese häufig verwendete Erweiterung zu LaTex. Die beiden Dokumente
sind über die Seminar-Seite
(\webaddress{http://www.mmi.rwth-aachen.de/Lehre/Seminare/Download})
verlinkt.  Weiterhin gibt es einige durchaus empfehlenswerte Bücher: 
\glqq\emph{\LaTeX: A Document Preparation System}\grqq, das Reference Manual \cite{latexManual} 
und \glqq\emph{The \LaTeX\ Companion}\grqq \cite{latexCompanion}, Erklärungen zu verschiedenen
Zusatzpackages u.\,a.\ \AmS-\LaTeX. Ausserdem gibt es noch die drei Latex
Bücher von Kopka, von denen nur das erste \glqq\emph{\LaTeX\ Einführung}\grqq
\cite{kopka1} für den einfachen Anwender von LaTex nötig ist.

\subsection{Überschriften und Referenzen}

\label{sub:CSheadlines}

Wie in diesem Beispieldokument zu sehen ist, gibt es für Artikel drei
Hierarchieebenen der Überschriften. Sie werden mit den Befehlen
\verb+\section{...}+, \verb+\subsection{...}+ und \verb+\subsubsection{...}+
gesetzt. Dabei ist \texttt{...} durch die jeweilig Überschrift zu
ersetzen. Innerhalb jedes Blocks kann durch den Befehl \texttt{label} eine
Marke für spätere Referenzierung gesetzt werden. In allen sochen Marken sollten
wiederum (wie bei Bilddateien) am Anfang stehen. Für diesen Abschnitt ist im
Quellcode der Befehl \verb+\label{sec:CSlatex}+ zu finden. Mit dem Befehl
\verb+\ref{sec:CSlatex}+ wird die Abschnittsnummer dieses Abschnitts
ausgegeben: 2. Für jede Überschriftenebene, sowie Formeln, Aufzählungspunkte,
Definitionen usw.\ gibt es eigene Zähler, die entsprechen mit label gesetzt
und mit ref wieder abgerufen werden können (s. Quelltext). Es zählt jeweils
die unterste Stufe. Daher ist es sinnvoll, den label Befehl direkt nach einem
einen neuen Block einleitenden Befehl zu verwenden.

Absätze werden in LaTeX durch eine oder mehrere Leerzeilen erreicht. Alle
überflüssigen Leerzeichen und -zeilen werden ignoriert. Befehle wie \verb+\\+
(newline), \verb+\clearpage+ oder \verb+\newpage+ sind nur in Sonderfällen zu
verwenden und in dieser Arbeit gar nicht.

\subsection{Textformatierung}
\label{sub:CSformat}

LaTeX bietet einige Möglichkeiten, die Schriftart zu verändern. Diese sollten
sparsam eingesetzt werden, um die Lesbarkeit des Textes nicht zu
beeinträchtigen. \emph{Wichtige} oder \emph{neue} Begriffe werden betont
(engl. emphasize) mit dem Befehl \verb+\emph{...}+. Die
Schreibmaschinenenschrift \verb+\texttt{...}+ wird für Befehle o\,ä.\ 
verwendet. Bei Befehlen, die LaTeX-Steuerzeichen enthalten empfiehlt sich der
Befehl \verb \verb+...+ \ (vgl. Kommentar im Quellcode in Abschnitt
\ref{sub:CStools}).

\subsection{Aufzählungen}
\label{sub:CSlists}

LaTeX stellt im wesentlichen zwei Arten von Aufzählungen zur Verfügung:
nummerierte und unnummerierte. Es handelt sich dabei um so genannte Umgebungen
(engl. environments), die immer von \verb+\begin{env}+ und \verb+\end{env}+
umschlossen werden. Dabei ist \texttt{env} durch den Namen der jeweiligen
Umgebung zu ersetzen. Wer Emacs verwendet kann mit \texttt{Ctrl-c Ctrl-e}
leicht Environments erstellen. Eine spezifische Aufzählungsumgebung für
Algorithmen wird in Abschnitt \ref{sub:CSmmiLists} erläutert. Es folgen
zwei Beispiele zu den beiden Aufzählungsarten.

\subsubsection{Itemize-Umgebung}
Dient der nicht nummerierten Aufzählung. Beispiel:
\begin{itemize}
\item Item1
\item Item2
\item Item3
\end{itemize}

\subsubsection{Enumerate-Umgebung}
Dient der nummerierten Aufzählung. Beispiel:
\begin{enumerate}
\item Item1
\item Item2
\item Item3 \label{enum:CSsubenum}
  \begin{enumerate}
  \item Item31
  \item Item32
  \item Item33
  \end{enumerate}
\end{enumerate}
Der \texttt{label}-Befehl funktioniert auch innerhalb von Aufzählungen 
(siehe Quellcode): Punkt \ref{enum:CSsubenum} ist in mehrere Unterpunkte aufgeteilt.

\subsection{Formel-Darstellung}
\label{sub:CSmath}

LaTex bietet zahlreiche komfortable Möglichkeiten, um Formeln darzustellen. 
Da aber die Darstellung sehr komplexer Formeln ermöglicht wird, existieren entsprechend viele 
Befehle zu ihrer Beschreibung. Zunächst muss man zwischen mathematischen
Ausdrücken im Text, wie $\alpha\in\mathbb{R}$, und abgesetzten Formeln oder
Ausdrücken entscheiden. Es folgen einige Beispiel aus anderen Texten, die
unter Zuhilfenahme des Quellcodes den Umgang mit Formeln verdeutlichen
sollten.

Hier zunächst zwei einfache Formeln.

\begin{equation}
  h_i(z)=\frac{1}{\sqrt{2\pi}\sigma_i}\cdot e^{-\left(\frac{z^2}{2\sigma_i^2}\right)}
\end{equation}
\begin{equation}
o_j=f_{\mathrm{sig}} \left( \sum_{i=1}^N c_{ij} h(||x-w_i||) \right)
\end{equation}

Mit \verb+\intertext{...}+ lässt sich Text innerhalb ausgerichteter
Formelblöcke einfügen.

\begin{align}
    w_i(t+1) &= w_i(t)-\alpha(t)\left[x(t)-w_i(t)\right]\nonumber \\ 
    w_j(t+1) &= w_j(t)+\alpha(t)\left[x(t)-w_j(t)\right]
  \label{eq:CSlvq2}
  \intertext{where $w_i$ must be of a different class than $x$ and $w_j$ of
    the same class. If the two closest prototypes are of the same class
    (training of type (2) the following modification scheme is applicable (no
    "window"\ constraint):}
  w_k(t+1) &= w_k(t) + \epsilon\alpha(t)\left[ x(t) - w_k(t)\right]
  \label{eq:CSlvq3}
\end{align}

\begin{equation}
\begin{split}
\label{eq:CScov}
\Sigma &= E\{(\mathbf{\mathbf{x}}-\mathbf{m})(\mathbf{\mathbf{x}}-\mathbf{m})^T\} \\
&= \left[ \begin{array}{ccc}
    E\{(x_1-m_1)(x_1-m_1)\} & \cdots & E\{(x_1-m_1)(x_n-m_n)\}\\
    \vdots & & \vdots \\
    E\{(x_n-m_n)(x_1-m_1)\} & \cdots & E\{(x_n-m_n)(x_n-m_n)\}\\
       \end{array} \right] \\
     &= \left[ \begin{array}{ccc}
         c_{11} & \cdots & c_{1n} \\
         \vdots & & \vdots \\
         c_{n1} & \cdots & c_{nn}
       \end{array} \right]
\end{split}
\end{equation}
Auch Formeln können natürlich Label besitzen, \zB\: Die Modifikation der Gewichte
beim LVQ3 Training \eqref{eq:CSlvq3} versteht ohne Erklärung keiner.


\section{Der mmiSeminar-Style}
\label{sec:CSstyle}

Der zunächst von Peter Dörfler für die Seminare am Lehrstuhl für Technische Informatik 
erstellte Style wird mit dem Wintersemester 06/07 auch für die Seminare des Instituts 
für Mensch-Maschine-Interaktion eingesetzt. Der Style bietet einige Makros, die das 
Erstellen von LaTeX Dokumenten vereinfachen. Durch Ändern einiger Verhaltensweisen 
von LaTeX wird es möglich, den Seminarband als ein TeX-Dokument zu behandeln. 
Weiterhin sind darin einheitliche Formatierungen und das Layout für den Text definiert.

\subsection{Die Titelseite}

\label{sub:CStitle}

Die Titelseite lässt sich recht schnell erstellen (siehe Quellcode). Es müssen
nur die folgenden Werte (eigentlich Makros) gesetzt werden:

\begin{narrowitems}
\item Der Autor des Textes
\item Der Titel
\item Die e-mail des Autors
\item Zusammenfassung
\item Schlüsselworte für den Inhalt
\end{narrowitems}
Der Befehl \verb+\makeArticleTitle+ erstellt dann die Titelseite. 

\subsection{Listen-Umgebungen im MMI-Style}
\label{sub:CSmmiLists}

\subsubsection{Aufzählungen}
Zwei Arten von Aufzählungen sind hinzugekommen:
\begin{narrowenum}
  \item \texttt{narrowenum}
  \item \texttt{narrowitems}
\end{narrowenum}
Die erste der beiden hat direkt hier Anwendung gefunden. Der andere
Aufzählungstyp ist in Abschnitt \ref{sub:CStitle} zu sehen. Diese Umgebungen
sind für Fälle gedacht, in denen die einzelnen Punkte sehr kurz sind. Die
normalen Abstände wirken einfach zu groß, wenn pro Punkt nur zwei, drei
Worte stehen.

\subsubsection{Algorithmus-Umgebung}

Die \texttt{algorithm}-Umgebung dient dazu Algorithmen
darzustellen. Sie funktioniert im Prinzip wie andere Listenumgebungen,
hat aber ein etwas anderes Layout. Zusätzlich sind alle weiteren Stufen nach
der ersten in die \texttt{subalgorithm}-Umgebung zu setzen. Üblicherweise wird
ein Algorithmus in eine \emph{floating}-Umgebung gesetzt. Das heisst, dass er
nicht unbedingt an der Stelle erscheint, an der der Algorithmus im Text steht, 
sondern eventuell in den nächsten freien Platz geschoben wird.

\begin{figure}[htb]
  \begin{algorithm}
  \item begin begin begin begin begin begin begin begin begin begin begin 
  			begin begin begin begin begin begin begin 
  \item \begin{subalgorithm}
    \item Item21 Item21 Item21 Item21 Item21 Item21 Item21 
    \item Item22 Item22 Item22 Item22 Item22 Item22 Item22 
    \end{subalgorithm}
  \item end
  \end{algorithm}
  \caption{Ein Beispiel-Algorithmus}
  \label{fig:CSalgo}
\end{figure}
Die \texttt{figure}-Umgebung (siehe Quellcode) ist für floating Bilder, mit der
Position als Parameter (\textbf{h}ere, \textbf{t}op, \textbf{b}ottom). Der
Befehl \texttt{caption} legt die Bildunterschrift fest. Diese Umgebung gehört
zu Standard-Latex.

\subsection{Bilder und Tabellen}
\label{sub:CSfigtab}

Sowohl für Tabellen als auch für Abbildungen gibt es im mmiSeminar-Style
floating-Umgebungen, die einfacher in der Anwendung sind als die
Standardumgebungen. 

\subsubsection{Abbildungen}

Abbildungen sollten im .jpg, .png oder .gif Format vorliegen und in der  \emph{figure}-Umgebung per \emph{includegraphics}-Befehl eingebunden werden. Die Bildgröße kann dabei auch relativ zur Textbreite angegeben werden, wie in Bild \ref{fig:Testbild} gezeigt.

\begin{figure}[htb]
 \centering \includegraphics[width=0.5\textwidth]{bilder/TestBild.png}
 \caption{Bildunterschrift für das Testbild. Sollte das Bild von einer anderen Quelle stammen wird hier auch mit cite darauf verwiesen.}
 \label{fig:Testbild}
\end{figure}

Auch in Abbildungen sollte es natürlich möglich sein, allen Text ohne Probleme
zu lesen. Leider sieht man oft das Gegenteil.

\subsubsection{Tabellen}

Die Tabellen-Umgebung \texttt{vtable} ist so ähnlich aufbebaut wie die
\texttt{xxxpsfig} Befehle weiter oben. Die Parameter sind in Reihenfolge:
\begin{narrowenum}
  \item Position
  \item Label
  \item Caption
  \item eigentliche Tabelle
\end{narrowenum}
Für die Labels gilt das gleich wie schon bei den Abbildungen. Der
Referenzierungsbefehl ist \texttt{tref}.  Die eigentliche Tabelle ist am
besten an einem Beispiel, Tabelle \tref{figs}, erklärt. Direkt nach dem Beginn
der \texttt{tabular}-Umgebung folgt ein Format-String, der die Ausrichtung des
Textes in den einzelnen Spalten festlegt. Die Optionen \texttt{l}, \texttt{r}
und \texttt{c} stehen für links, rechts und mittig ausgerichtet. Mit
\texttt{p\{Länge\}} kann man eine Spalte fester Breite deklarieren. Der
senkrechte Strich erzeugt einen ebensolchen. Vom Layout gefälliger und auch
lesbarer ist die Verwendung nur nötiger Linien. Die im Beispiel wäre \zB\ auch
nicht nötig gewesen. Mit \texttt{multiline\{spalten\}\{ausrichtung\}\{text\}}
kann man einen Text über mehrere Spalten ziehen und ausrichten wie angegeben.
Alles weitere steht in der angegebenen Literatur.

\appendixOfArticle
\section{Testing the Appendix}

Der Anhang wird durch den Befehl \texttt{appendixOfArticle} eingeleitet. Auf
keinen Fall ist der Befehl \texttt{appendix} zu verwenden. 

Im Anhang stehen solche Dinge, die nicht in den sonstigen Text passen. Lange
Beweise und zusätzliche Beispiele oder Abbildungen gehören dazu. Im
Allgemeinen sollte ein Anhang möglichst ganz vermieden werden.

%%% Local Variables: 
%%% mode: latex
%%% TeX-master: "frame"
%%% End: 







+ fügt dieses Dokument (MyText.tex) ein. \glqq
  MyText\grqq\ ist durch den Namen der eigenen Datei zu ersetzen. Weiter unter
  wird mit dem Befehl \verb+\bibliography{MyBib}+ festgelegt, dass die
  Literaturhinweise in der Datei \glqq MyBib.bib\grqq\ stehen. Diese sollte
  bei den Seminararbeiten den gleichen Namen wie der Text haben.
  
\item \texttt{MyText.tex}\\
  Der Quelltext dieser Anleitung. Neben dieser Funktion dient die Datei auch
  als Vorlage für die Seminararbeiten. Mit Hilfe der PostScript Version und
  des Quellcodes können die Befehle um ein bestimmtes Layout zu erhalten
  leicht gefunden werden. Die eigene Arbeit soll den Namen
  \texttt{nachname.tex} haben (alles klein).
  
\item \texttt{MyBib.bib}\\
  Enthält die Literaturangaben zu diesem Text. Die Datei enthält weiterhin
  Vorlagen für andere Typen von Literaturangaben für die Nicht-Emacs-User. Das
  eigene Literaturverzeichnis soll den Namen \texttt{nachname.bib} haben
  (alles klein).
  
\item \texttt{xxx.tex xxx.sty xxx.bst}\\
  In diesen Dateien werden Styles und Makros definiert, die Erweiterungen oder
  Veränderungen von LaTeX darstellen. Sie dürfen auf keinen Fall verändert
  werden. 
  
\end{itemize}

Weiterhin ist sehr wichtig, dass alle Bilder als .jpg, .png oder .gif gespeichert sind und
sich im Unterverzeichnis \texttt{bilder} relativ zu eurem Hauptfile befinden. 
Wenn also euer \texttt{nachname.tex} in \texttt{home/username/seminar steht}, gehören
die Bilder in \texttt{home/username/seminar/bilder}. 

\section{Sehr kurze Einführung in \LaTeX}

\label{sec:CSlatex}

Für komplexere Aufgaben sind weitere Nachschlagewerke unverzichtbar. 
Eine recht umfassende Beschreibung ist kostenlos erhältlich:
\glqq\emph{The Not So Short Introduction to \LaTeX2e}\grqq \cite{oetiker},
eine Einführung in Latex. Für schwierige Formeln ist das \AmS-\LaTeX-Package
sehr zu empfehlen. Das \glqq\emph{\AmS-\LaTeX\ User's Guide} \cite{ams}
beschreibt diese häufig verwendete Erweiterung zu LaTex. Die beiden Dokumente
sind über die Seminar-Seite
(\webaddress{http://www.mmi.rwth-aachen.de/Lehre/Seminare/Download})
verlinkt.  Weiterhin gibt es einige durchaus empfehlenswerte Bücher: 
\glqq\emph{\LaTeX: A Document Preparation System}\grqq, das Reference Manual \cite{latexManual} 
und \glqq\emph{The \LaTeX\ Companion}\grqq \cite{latexCompanion}, Erklärungen zu verschiedenen
Zusatzpackages u.\,a.\ \AmS-\LaTeX. Ausserdem gibt es noch die drei Latex
Bücher von Kopka, von denen nur das erste \glqq\emph{\LaTeX\ Einführung}\grqq
\cite{kopka1} für den einfachen Anwender von LaTex nötig ist.

\subsection{Überschriften und Referenzen}

\label{sub:CSheadlines}

Wie in diesem Beispieldokument zu sehen ist, gibt es für Artikel drei
Hierarchieebenen der Überschriften. Sie werden mit den Befehlen
\verb+\section{...}+, \verb+\subsection{...}+ und \verb+\subsubsection{...}+
gesetzt. Dabei ist \texttt{...} durch die jeweilig Überschrift zu
ersetzen. Innerhalb jedes Blocks kann durch den Befehl \texttt{label} eine
Marke für spätere Referenzierung gesetzt werden. In allen sochen Marken sollten
wiederum (wie bei Bilddateien) am Anfang stehen. Für diesen Abschnitt ist im
Quellcode der Befehl \verb+\label{sec:CSlatex}+ zu finden. Mit dem Befehl
\verb+\ref{sec:CSlatex}+ wird die Abschnittsnummer dieses Abschnitts
ausgegeben: 2. Für jede Überschriftenebene, sowie Formeln, Aufzählungspunkte,
Definitionen usw.\ gibt es eigene Zähler, die entsprechen mit label gesetzt
und mit ref wieder abgerufen werden können (s. Quelltext). Es zählt jeweils
die unterste Stufe. Daher ist es sinnvoll, den label Befehl direkt nach einem
einen neuen Block einleitenden Befehl zu verwenden.

Absätze werden in LaTeX durch eine oder mehrere Leerzeilen erreicht. Alle
überflüssigen Leerzeichen und -zeilen werden ignoriert. Befehle wie \verb+\\+
(newline), \verb+\clearpage+ oder \verb+\newpage+ sind nur in Sonderfällen zu
verwenden und in dieser Arbeit gar nicht.

\subsection{Textformatierung}
\label{sub:CSformat}

LaTeX bietet einige Möglichkeiten, die Schriftart zu verändern. Diese sollten
sparsam eingesetzt werden, um die Lesbarkeit des Textes nicht zu
beeinträchtigen. \emph{Wichtige} oder \emph{neue} Begriffe werden betont
(engl. emphasize) mit dem Befehl \verb+\emph{...}+. Die
Schreibmaschinenenschrift \verb+\texttt{...}+ wird für Befehle o\,ä.\ 
verwendet. Bei Befehlen, die LaTeX-Steuerzeichen enthalten empfiehlt sich der
Befehl \verb \verb+...+ \ (vgl. Kommentar im Quellcode in Abschnitt
\ref{sub:CStools}).

\subsection{Aufzählungen}
\label{sub:CSlists}

LaTeX stellt im wesentlichen zwei Arten von Aufzählungen zur Verfügung:
nummerierte und unnummerierte. Es handelt sich dabei um so genannte Umgebungen
(engl. environments), die immer von \verb+\begin{env}+ und \verb+\end{env}+
umschlossen werden. Dabei ist \texttt{env} durch den Namen der jeweiligen
Umgebung zu ersetzen. Wer Emacs verwendet kann mit \texttt{Ctrl-c Ctrl-e}
leicht Environments erstellen. Eine spezifische Aufzählungsumgebung für
Algorithmen wird in Abschnitt \ref{sub:CSmmiLists} erläutert. Es folgen
zwei Beispiele zu den beiden Aufzählungsarten.

\subsubsection{Itemize-Umgebung}
Dient der nicht nummerierten Aufzählung. Beispiel:
\begin{itemize}
\item Item1
\item Item2
\item Item3
\end{itemize}

\subsubsection{Enumerate-Umgebung}
Dient der nummerierten Aufzählung. Beispiel:
\begin{enumerate}
\item Item1
\item Item2
\item Item3 \label{enum:CSsubenum}
  \begin{enumerate}
  \item Item31
  \item Item32
  \item Item33
  \end{enumerate}
\end{enumerate}
Der \texttt{label}-Befehl funktioniert auch innerhalb von Aufzählungen 
(siehe Quellcode): Punkt \ref{enum:CSsubenum} ist in mehrere Unterpunkte aufgeteilt.

\subsection{Formel-Darstellung}
\label{sub:CSmath}

LaTex bietet zahlreiche komfortable Möglichkeiten, um Formeln darzustellen. 
Da aber die Darstellung sehr komplexer Formeln ermöglicht wird, existieren entsprechend viele 
Befehle zu ihrer Beschreibung. Zunächst muss man zwischen mathematischen
Ausdrücken im Text, wie $\alpha\in\mathbb{R}$, und abgesetzten Formeln oder
Ausdrücken entscheiden. Es folgen einige Beispiel aus anderen Texten, die
unter Zuhilfenahme des Quellcodes den Umgang mit Formeln verdeutlichen
sollten.

Hier zunächst zwei einfache Formeln.

\begin{equation}
  h_i(z)=\frac{1}{\sqrt{2\pi}\sigma_i}\cdot e^{-\left(\frac{z^2}{2\sigma_i^2}\right)}
\end{equation}
\begin{equation}
o_j=f_{\mathrm{sig}} \left( \sum_{i=1}^N c_{ij} h(||x-w_i||) \right)
\end{equation}

Mit \verb+\intertext{...}+ lässt sich Text innerhalb ausgerichteter
Formelblöcke einfügen.

\begin{align}
    w_i(t+1) &= w_i(t)-\alpha(t)\left[x(t)-w_i(t)\right]\nonumber \\ 
    w_j(t+1) &= w_j(t)+\alpha(t)\left[x(t)-w_j(t)\right]
  \label{eq:CSlvq2}
  \intertext{where $w_i$ must be of a different class than $x$ and $w_j$ of
    the same class. If the two closest prototypes are of the same class
    (training of type (2) the following modification scheme is applicable (no
    "window"\ constraint):}
  w_k(t+1) &= w_k(t) + \epsilon\alpha(t)\left[ x(t) - w_k(t)\right]
  \label{eq:CSlvq3}
\end{align}

\begin{equation}
\begin{split}
\label{eq:CScov}
\Sigma &= E\{(\mathbf{\mathbf{x}}-\mathbf{m})(\mathbf{\mathbf{x}}-\mathbf{m})^T\} \\
&= \left[ \begin{array}{ccc}
    E\{(x_1-m_1)(x_1-m_1)\} & \cdots & E\{(x_1-m_1)(x_n-m_n)\}\\
    \vdots & & \vdots \\
    E\{(x_n-m_n)(x_1-m_1)\} & \cdots & E\{(x_n-m_n)(x_n-m_n)\}\\
       \end{array} \right] \\
     &= \left[ \begin{array}{ccc}
         c_{11} & \cdots & c_{1n} \\
         \vdots & & \vdots \\
         c_{n1} & \cdots & c_{nn}
       \end{array} \right]
\end{split}
\end{equation}
Auch Formeln können natürlich Label besitzen, \zB\: Die Modifikation der Gewichte
beim LVQ3 Training \eqref{eq:CSlvq3} versteht ohne Erklärung keiner.


\section{Der mmiSeminar-Style}
\label{sec:CSstyle}

Der zunächst von Peter Dörfler für die Seminare am Lehrstuhl für Technische Informatik 
erstellte Style wird mit dem Wintersemester 06/07 auch für die Seminare des Instituts 
für Mensch-Maschine-Interaktion eingesetzt. Der Style bietet einige Makros, die das 
Erstellen von LaTeX Dokumenten vereinfachen. Durch Ändern einiger Verhaltensweisen 
von LaTeX wird es möglich, den Seminarband als ein TeX-Dokument zu behandeln. 
Weiterhin sind darin einheitliche Formatierungen und das Layout für den Text definiert.

\subsection{Die Titelseite}

\label{sub:CStitle}

Die Titelseite lässt sich recht schnell erstellen (siehe Quellcode). Es müssen
nur die folgenden Werte (eigentlich Makros) gesetzt werden:

\begin{narrowitems}
\item Der Autor des Textes
\item Der Titel
\item Die e-mail des Autors
\item Zusammenfassung
\item Schlüsselworte für den Inhalt
\end{narrowitems}
Der Befehl \verb+\makeArticleTitle+ erstellt dann die Titelseite. 

\subsection{Listen-Umgebungen im MMI-Style}
\label{sub:CSmmiLists}

\subsubsection{Aufzählungen}
Zwei Arten von Aufzählungen sind hinzugekommen:
\begin{narrowenum}
  \item \texttt{narrowenum}
  \item \texttt{narrowitems}
\end{narrowenum}
Die erste der beiden hat direkt hier Anwendung gefunden. Der andere
Aufzählungstyp ist in Abschnitt \ref{sub:CStitle} zu sehen. Diese Umgebungen
sind für Fälle gedacht, in denen die einzelnen Punkte sehr kurz sind. Die
normalen Abstände wirken einfach zu groß, wenn pro Punkt nur zwei, drei
Worte stehen.

\subsubsection{Algorithmus-Umgebung}

Die \texttt{algorithm}-Umgebung dient dazu Algorithmen
darzustellen. Sie funktioniert im Prinzip wie andere Listenumgebungen,
hat aber ein etwas anderes Layout. Zusätzlich sind alle weiteren Stufen nach
der ersten in die \texttt{subalgorithm}-Umgebung zu setzen. Üblicherweise wird
ein Algorithmus in eine \emph{floating}-Umgebung gesetzt. Das heisst, dass er
nicht unbedingt an der Stelle erscheint, an der der Algorithmus im Text steht, 
sondern eventuell in den nächsten freien Platz geschoben wird.

\begin{figure}[htb]
  \begin{algorithm}
  \item begin begin begin begin begin begin begin begin begin begin begin 
  			begin begin begin begin begin begin begin 
  \item \begin{subalgorithm}
    \item Item21 Item21 Item21 Item21 Item21 Item21 Item21 
    \item Item22 Item22 Item22 Item22 Item22 Item22 Item22 
    \end{subalgorithm}
  \item end
  \end{algorithm}
  \caption{Ein Beispiel-Algorithmus}
  \label{fig:CSalgo}
\end{figure}
Die \texttt{figure}-Umgebung (siehe Quellcode) ist für floating Bilder, mit der
Position als Parameter (\textbf{h}ere, \textbf{t}op, \textbf{b}ottom). Der
Befehl \texttt{caption} legt die Bildunterschrift fest. Diese Umgebung gehört
zu Standard-Latex.

\subsection{Bilder und Tabellen}
\label{sub:CSfigtab}

Sowohl für Tabellen als auch für Abbildungen gibt es im mmiSeminar-Style
floating-Umgebungen, die einfacher in der Anwendung sind als die
Standardumgebungen. 

\subsubsection{Abbildungen}

Abbildungen sollten im .jpg, .png oder .gif Format vorliegen und in der  \emph{figure}-Umgebung per \emph{includegraphics}-Befehl eingebunden werden. Die Bildgröße kann dabei auch relativ zur Textbreite angegeben werden, wie in Bild \ref{fig:Testbild} gezeigt.

\begin{figure}[htb]
 \centering \includegraphics[width=0.5\textwidth]{bilder/TestBild.png}
 \caption{Bildunterschrift für das Testbild. Sollte das Bild von einer anderen Quelle stammen wird hier auch mit cite darauf verwiesen.}
 \label{fig:Testbild}
\end{figure}

Auch in Abbildungen sollte es natürlich möglich sein, allen Text ohne Probleme
zu lesen. Leider sieht man oft das Gegenteil.

\subsubsection{Tabellen}

Die Tabellen-Umgebung \texttt{vtable} ist so ähnlich aufbebaut wie die
\texttt{xxxpsfig} Befehle weiter oben. Die Parameter sind in Reihenfolge:
\begin{narrowenum}
  \item Position
  \item Label
  \item Caption
  \item eigentliche Tabelle
\end{narrowenum}
Für die Labels gilt das gleich wie schon bei den Abbildungen. Der
Referenzierungsbefehl ist \texttt{tref}.  Die eigentliche Tabelle ist am
besten an einem Beispiel, Tabelle \tref{figs}, erklärt. Direkt nach dem Beginn
der \texttt{tabular}-Umgebung folgt ein Format-String, der die Ausrichtung des
Textes in den einzelnen Spalten festlegt. Die Optionen \texttt{l}, \texttt{r}
und \texttt{c} stehen für links, rechts und mittig ausgerichtet. Mit
\texttt{p\{Länge\}} kann man eine Spalte fester Breite deklarieren. Der
senkrechte Strich erzeugt einen ebensolchen. Vom Layout gefälliger und auch
lesbarer ist die Verwendung nur nötiger Linien. Die im Beispiel wäre \zB\ auch
nicht nötig gewesen. Mit \texttt{multiline\{spalten\}\{ausrichtung\}\{text\}}
kann man einen Text über mehrere Spalten ziehen und ausrichten wie angegeben.
Alles weitere steht in der angegebenen Literatur.

\appendixOfArticle
\section{Testing the Appendix}

Der Anhang wird durch den Befehl \texttt{appendixOfArticle} eingeleitet. Auf
keinen Fall ist der Befehl \texttt{appendix} zu verwenden. 

Im Anhang stehen solche Dinge, die nicht in den sonstigen Text passen. Lange
Beweise und zusätzliche Beispiele oder Abbildungen gehören dazu. Im
Allgemeinen sollte ein Anhang möglichst ganz vermieden werden.

%%% Local Variables: 
%%% mode: latex
%%% TeX-master: "frame"
%%% End: 







+ fügt dieses Dokument (MyText.tex) ein. \glqq
  MyText\grqq\ ist durch den Namen der eigenen Datei zu ersetzen. Weiter unter
  wird mit dem Befehl \verb+\bibliography{MyBib}+ festgelegt, dass die
  Literaturhinweise in der Datei \glqq MyBib.bib\grqq\ stehen. Diese sollte
  bei den Seminararbeiten den gleichen Namen wie der Text haben.
  
\item \texttt{MyText.tex}\\
  Der Quelltext dieser Anleitung. Neben dieser Funktion dient die Datei auch
  als Vorlage für die Seminararbeiten. Mit Hilfe der PostScript Version und
  des Quellcodes können die Befehle um ein bestimmtes Layout zu erhalten
  leicht gefunden werden. Die eigene Arbeit soll den Namen
  \texttt{nachname.tex} haben (alles klein).
  
\item \texttt{MyBib.bib}\\
  Enthält die Literaturangaben zu diesem Text. Die Datei enthält weiterhin
  Vorlagen für andere Typen von Literaturangaben für die Nicht-Emacs-User. Das
  eigene Literaturverzeichnis soll den Namen \texttt{nachname.bib} haben
  (alles klein).
  
\item \texttt{xxx.tex xxx.sty xxx.bst}\\
  In diesen Dateien werden Styles und Makros definiert, die Erweiterungen oder
  Veränderungen von LaTeX darstellen. Sie dürfen auf keinen Fall verändert
  werden. 
  
\end{itemize}

Weiterhin ist sehr wichtig, dass alle Bilder als .jpg, .png oder .gif gespeichert sind und
sich im Unterverzeichnis \texttt{bilder} relativ zu eurem Hauptfile befinden. 
Wenn also euer \texttt{nachname.tex} in \texttt{home/username/seminar steht}, gehören
die Bilder in \texttt{home/username/seminar/bilder}. 

\section{Sehr kurze Einführung in \LaTeX}

\label{sec:CSlatex}

Für komplexere Aufgaben sind weitere Nachschlagewerke unverzichtbar. 
Eine recht umfassende Beschreibung ist kostenlos erhältlich:
\glqq\emph{The Not So Short Introduction to \LaTeX2e}\grqq \cite{oetiker},
eine Einführung in Latex. Für schwierige Formeln ist das \AmS-\LaTeX-Package
sehr zu empfehlen. Das \glqq\emph{\AmS-\LaTeX\ User's Guide} \cite{ams}
beschreibt diese häufig verwendete Erweiterung zu LaTex. Die beiden Dokumente
sind über die Seminar-Seite
(\webaddress{http://www.mmi.rwth-aachen.de/Lehre/Seminare/Download})
verlinkt.  Weiterhin gibt es einige durchaus empfehlenswerte Bücher: 
\glqq\emph{\LaTeX: A Document Preparation System}\grqq, das Reference Manual \cite{latexManual} 
und \glqq\emph{The \LaTeX\ Companion}\grqq \cite{latexCompanion}, Erklärungen zu verschiedenen
Zusatzpackages u.\,a.\ \AmS-\LaTeX. Ausserdem gibt es noch die drei Latex
Bücher von Kopka, von denen nur das erste \glqq\emph{\LaTeX\ Einführung}\grqq
\cite{kopka1} für den einfachen Anwender von LaTex nötig ist.

\subsection{Überschriften und Referenzen}

\label{sub:CSheadlines}

Wie in diesem Beispieldokument zu sehen ist, gibt es für Artikel drei
Hierarchieebenen der Überschriften. Sie werden mit den Befehlen
\verb+\section{...}+, \verb+\subsection{...}+ und \verb+\subsubsection{...}+
gesetzt. Dabei ist \texttt{...} durch die jeweilig Überschrift zu
ersetzen. Innerhalb jedes Blocks kann durch den Befehl \texttt{label} eine
Marke für spätere Referenzierung gesetzt werden. In allen sochen Marken sollten
wiederum (wie bei Bilddateien) am Anfang stehen. Für diesen Abschnitt ist im
Quellcode der Befehl \verb+\label{sec:CSlatex}+ zu finden. Mit dem Befehl
\verb+\ref{sec:CSlatex}+ wird die Abschnittsnummer dieses Abschnitts
ausgegeben: 2. Für jede Überschriftenebene, sowie Formeln, Aufzählungspunkte,
Definitionen usw.\ gibt es eigene Zähler, die entsprechen mit label gesetzt
und mit ref wieder abgerufen werden können (s. Quelltext). Es zählt jeweils
die unterste Stufe. Daher ist es sinnvoll, den label Befehl direkt nach einem
einen neuen Block einleitenden Befehl zu verwenden.

Absätze werden in LaTeX durch eine oder mehrere Leerzeilen erreicht. Alle
überflüssigen Leerzeichen und -zeilen werden ignoriert. Befehle wie \verb+\\+
(newline), \verb+\clearpage+ oder \verb+\newpage+ sind nur in Sonderfällen zu
verwenden und in dieser Arbeit gar nicht.

\subsection{Textformatierung}
\label{sub:CSformat}

LaTeX bietet einige Möglichkeiten, die Schriftart zu verändern. Diese sollten
sparsam eingesetzt werden, um die Lesbarkeit des Textes nicht zu
beeinträchtigen. \emph{Wichtige} oder \emph{neue} Begriffe werden betont
(engl. emphasize) mit dem Befehl \verb+\emph{...}+. Die
Schreibmaschinenenschrift \verb+\texttt{...}+ wird für Befehle o\,ä.\ 
verwendet. Bei Befehlen, die LaTeX-Steuerzeichen enthalten empfiehlt sich der
Befehl \verb \verb+...+ \ (vgl. Kommentar im Quellcode in Abschnitt
\ref{sub:CStools}).

\subsection{Aufzählungen}
\label{sub:CSlists}

LaTeX stellt im wesentlichen zwei Arten von Aufzählungen zur Verfügung:
nummerierte und unnummerierte. Es handelt sich dabei um so genannte Umgebungen
(engl. environments), die immer von \verb+\begin{env}+ und \verb+\end{env}+
umschlossen werden. Dabei ist \texttt{env} durch den Namen der jeweiligen
Umgebung zu ersetzen. Wer Emacs verwendet kann mit \texttt{Ctrl-c Ctrl-e}
leicht Environments erstellen. Eine spezifische Aufzählungsumgebung für
Algorithmen wird in Abschnitt \ref{sub:CSmmiLists} erläutert. Es folgen
zwei Beispiele zu den beiden Aufzählungsarten.

\subsubsection{Itemize-Umgebung}
Dient der nicht nummerierten Aufzählung. Beispiel:
\begin{itemize}
\item Item1
\item Item2
\item Item3
\end{itemize}

\subsubsection{Enumerate-Umgebung}
Dient der nummerierten Aufzählung. Beispiel:
\begin{enumerate}
\item Item1
\item Item2
\item Item3 \label{enum:CSsubenum}
  \begin{enumerate}
  \item Item31
  \item Item32
  \item Item33
  \end{enumerate}
\end{enumerate}
Der \texttt{label}-Befehl funktioniert auch innerhalb von Aufzählungen 
(siehe Quellcode): Punkt \ref{enum:CSsubenum} ist in mehrere Unterpunkte aufgeteilt.

\subsection{Formel-Darstellung}
\label{sub:CSmath}

LaTex bietet zahlreiche komfortable Möglichkeiten, um Formeln darzustellen. 
Da aber die Darstellung sehr komplexer Formeln ermöglicht wird, existieren entsprechend viele 
Befehle zu ihrer Beschreibung. Zunächst muss man zwischen mathematischen
Ausdrücken im Text, wie $\alpha\in\mathbb{R}$, und abgesetzten Formeln oder
Ausdrücken entscheiden. Es folgen einige Beispiel aus anderen Texten, die
unter Zuhilfenahme des Quellcodes den Umgang mit Formeln verdeutlichen
sollten.

Hier zunächst zwei einfache Formeln.

\begin{equation}
  h_i(z)=\frac{1}{\sqrt{2\pi}\sigma_i}\cdot e^{-\left(\frac{z^2}{2\sigma_i^2}\right)}
\end{equation}
\begin{equation}
o_j=f_{\mathrm{sig}} \left( \sum_{i=1}^N c_{ij} h(||x-w_i||) \right)
\end{equation}

Mit \verb+\intertext{...}+ lässt sich Text innerhalb ausgerichteter
Formelblöcke einfügen.

\begin{align}
    w_i(t+1) &= w_i(t)-\alpha(t)\left[x(t)-w_i(t)\right]\nonumber \\ 
    w_j(t+1) &= w_j(t)+\alpha(t)\left[x(t)-w_j(t)\right]
  \label{eq:CSlvq2}
  \intertext{where $w_i$ must be of a different class than $x$ and $w_j$ of
    the same class. If the two closest prototypes are of the same class
    (training of type (2) the following modification scheme is applicable (no
    "window"\ constraint):}
  w_k(t+1) &= w_k(t) + \epsilon\alpha(t)\left[ x(t) - w_k(t)\right]
  \label{eq:CSlvq3}
\end{align}

\begin{equation}
\begin{split}
\label{eq:CScov}
\Sigma &= E\{(\mathbf{\mathbf{x}}-\mathbf{m})(\mathbf{\mathbf{x}}-\mathbf{m})^T\} \\
&= \left[ \begin{array}{ccc}
    E\{(x_1-m_1)(x_1-m_1)\} & \cdots & E\{(x_1-m_1)(x_n-m_n)\}\\
    \vdots & & \vdots \\
    E\{(x_n-m_n)(x_1-m_1)\} & \cdots & E\{(x_n-m_n)(x_n-m_n)\}\\
       \end{array} \right] \\
     &= \left[ \begin{array}{ccc}
         c_{11} & \cdots & c_{1n} \\
         \vdots & & \vdots \\
         c_{n1} & \cdots & c_{nn}
       \end{array} \right]
\end{split}
\end{equation}
Auch Formeln können natürlich Label besitzen, \zB\: Die Modifikation der Gewichte
beim LVQ3 Training \eqref{eq:CSlvq3} versteht ohne Erklärung keiner.


\section{Der mmiSeminar-Style}
\label{sec:CSstyle}

Der zunächst von Peter Dörfler für die Seminare am Lehrstuhl für Technische Informatik 
erstellte Style wird mit dem Wintersemester 06/07 auch für die Seminare des Instituts 
für Mensch-Maschine-Interaktion eingesetzt. Der Style bietet einige Makros, die das 
Erstellen von LaTeX Dokumenten vereinfachen. Durch Ändern einiger Verhaltensweisen 
von LaTeX wird es möglich, den Seminarband als ein TeX-Dokument zu behandeln. 
Weiterhin sind darin einheitliche Formatierungen und das Layout für den Text definiert.

\subsection{Die Titelseite}

\label{sub:CStitle}

Die Titelseite lässt sich recht schnell erstellen (siehe Quellcode). Es müssen
nur die folgenden Werte (eigentlich Makros) gesetzt werden:

\begin{narrowitems}
\item Der Autor des Textes
\item Der Titel
\item Die e-mail des Autors
\item Zusammenfassung
\item Schlüsselworte für den Inhalt
\end{narrowitems}
Der Befehl \verb+\makeArticleTitle+ erstellt dann die Titelseite. 

\subsection{Listen-Umgebungen im MMI-Style}
\label{sub:CSmmiLists}

\subsubsection{Aufzählungen}
Zwei Arten von Aufzählungen sind hinzugekommen:
\begin{narrowenum}
  \item \texttt{narrowenum}
  \item \texttt{narrowitems}
\end{narrowenum}
Die erste der beiden hat direkt hier Anwendung gefunden. Der andere
Aufzählungstyp ist in Abschnitt \ref{sub:CStitle} zu sehen. Diese Umgebungen
sind für Fälle gedacht, in denen die einzelnen Punkte sehr kurz sind. Die
normalen Abstände wirken einfach zu groß, wenn pro Punkt nur zwei, drei
Worte stehen.

\subsubsection{Algorithmus-Umgebung}

Die \texttt{algorithm}-Umgebung dient dazu Algorithmen
darzustellen. Sie funktioniert im Prinzip wie andere Listenumgebungen,
hat aber ein etwas anderes Layout. Zusätzlich sind alle weiteren Stufen nach
der ersten in die \texttt{subalgorithm}-Umgebung zu setzen. Üblicherweise wird
ein Algorithmus in eine \emph{floating}-Umgebung gesetzt. Das heisst, dass er
nicht unbedingt an der Stelle erscheint, an der der Algorithmus im Text steht, 
sondern eventuell in den nächsten freien Platz geschoben wird.

\begin{figure}[htb]
  \begin{algorithm}
  \item begin begin begin begin begin begin begin begin begin begin begin 
  			begin begin begin begin begin begin begin 
  \item \begin{subalgorithm}
    \item Item21 Item21 Item21 Item21 Item21 Item21 Item21 
    \item Item22 Item22 Item22 Item22 Item22 Item22 Item22 
    \end{subalgorithm}
  \item end
  \end{algorithm}
  \caption{Ein Beispiel-Algorithmus}
  \label{fig:CSalgo}
\end{figure}
Die \texttt{figure}-Umgebung (siehe Quellcode) ist für floating Bilder, mit der
Position als Parameter (\textbf{h}ere, \textbf{t}op, \textbf{b}ottom). Der
Befehl \texttt{caption} legt die Bildunterschrift fest. Diese Umgebung gehört
zu Standard-Latex.

\subsection{Bilder und Tabellen}
\label{sub:CSfigtab}

Sowohl für Tabellen als auch für Abbildungen gibt es im mmiSeminar-Style
floating-Umgebungen, die einfacher in der Anwendung sind als die
Standardumgebungen. 

\subsubsection{Abbildungen}

Abbildungen sollten im .jpg, .png oder .gif Format vorliegen und in der  \emph{figure}-Umgebung per \emph{includegraphics}-Befehl eingebunden werden. Die Bildgröße kann dabei auch relativ zur Textbreite angegeben werden, wie in Bild \ref{fig:Testbild} gezeigt.

\begin{figure}[htb]
 \centering \includegraphics[width=0.5\textwidth]{bilder/TestBild.png}
 \caption{Bildunterschrift für das Testbild. Sollte das Bild von einer anderen Quelle stammen wird hier auch mit cite darauf verwiesen.}
 \label{fig:Testbild}
\end{figure}

Auch in Abbildungen sollte es natürlich möglich sein, allen Text ohne Probleme
zu lesen. Leider sieht man oft das Gegenteil.

\subsubsection{Tabellen}

Die Tabellen-Umgebung \texttt{vtable} ist so ähnlich aufbebaut wie die
\texttt{xxxpsfig} Befehle weiter oben. Die Parameter sind in Reihenfolge:
\begin{narrowenum}
  \item Position
  \item Label
  \item Caption
  \item eigentliche Tabelle
\end{narrowenum}
Für die Labels gilt das gleich wie schon bei den Abbildungen. Der
Referenzierungsbefehl ist \texttt{tref}.  Die eigentliche Tabelle ist am
besten an einem Beispiel, Tabelle \tref{figs}, erklärt. Direkt nach dem Beginn
der \texttt{tabular}-Umgebung folgt ein Format-String, der die Ausrichtung des
Textes in den einzelnen Spalten festlegt. Die Optionen \texttt{l}, \texttt{r}
und \texttt{c} stehen für links, rechts und mittig ausgerichtet. Mit
\texttt{p\{Länge\}} kann man eine Spalte fester Breite deklarieren. Der
senkrechte Strich erzeugt einen ebensolchen. Vom Layout gefälliger und auch
lesbarer ist die Verwendung nur nötiger Linien. Die im Beispiel wäre \zB\ auch
nicht nötig gewesen. Mit \texttt{multiline\{spalten\}\{ausrichtung\}\{text\}}
kann man einen Text über mehrere Spalten ziehen und ausrichten wie angegeben.
Alles weitere steht in der angegebenen Literatur.

\appendixOfArticle
\section{Testing the Appendix}

Der Anhang wird durch den Befehl \texttt{appendixOfArticle} eingeleitet. Auf
keinen Fall ist der Befehl \texttt{appendix} zu verwenden. 

Im Anhang stehen solche Dinge, die nicht in den sonstigen Text passen. Lange
Beweise und zusätzliche Beispiele oder Abbildungen gehören dazu. Im
Allgemeinen sollte ein Anhang möglichst ganz vermieden werden.

%%% Local Variables: 
%%% mode: latex
%%% TeX-master: "frame"
%%% End: 







